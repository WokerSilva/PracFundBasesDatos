\documentclass[a4paper,12pt]{article} 
\usepackage[utf8]{inputenc} % Acentos válidos sin problemas
\usepackage[spanish]{babel} % Idioma
%\usepackage[style=biber]{biblatex}
%\addbibresource{bibliografia.bib}
%\usepackage[
%  backend=biber
%]{biblatex}
\usepackage[backend=biber, style=ieee]{biblatex}

\addbibresource{bibliografia.bib}
\usepackage{csquotes}
%\usepackage{background}
%\setlegtth{\parindent}{2px}
%\phantom{abc}

%-----------------------------------INICIO DE PACKETES------------------/
%----------------------------------------------------------------------/|
\usepackage{amsmath}   % Matemáticas: Comandos extras(cajas ecuaciones) |
\usepackage{amssymb}   % Matemáticas: Símbolos matemáticos              |
\usepackage{datetime}  % Agregar fechas                                 |
\usepackage{graphicx}  % Insertar Imágenes                              |
\usepackage{biblatex} % Bibliografía                                   |
\usepackage{multicol}  % Creación de tablas                             |
\usepackage{longtable} % Tablas más largas                              |
\usepackage{xcolor}    % Permite cambiar colores del texto              |
\usepackage{tcolorbox} % Cajas de color                                 |
\usepackage{setspace}  % Usar espacios                                  |
\usepackage{fancyhdr}  % Para agregar encabezado y pie de página        |
\usepackage{lastpage}  %                                                 |
\usepackage{float}     % Flotantes                                      |
\usepackage{soul}      % "Efectos" en palabras                          |
\usepackage{hyperref}  % Para usar hipervínculos                        |
\usepackage{caption}   % Utilizar las referencias                       |
\usepackage{subcaption} % Poder usar subfiguras                         |
\usepackage{multirow}  % Nos permite modificar tablas                   |
\usepackage{array}     % Permite utilizar los valores para multicolumn  |
\usepackage{booktabs}  % Permite modificar tablas                       |
\usepackage{diagbox}   % Diagonales para las tablas                     |
\usepackage{colortbl}  % Color para tablas                              |
%\usepackage{listings}  % Escribir código                                |
\usepackage{mathtools} % SIMBOLO :=                                     |
\usepackage{enumitem}  % Modificar items de Listas                      |
\usepackage{tikz}      %                                                |
\usepackage{lipsum}    % for auto generating text                       |
\usepackage{afterpage} % for "\afterpage"s                              |
\usepackage{pagecolor} % With option pagecolor={somecolor or none}|     |
\usepackage{xpatch}    % Color de lineas C & F
%\usepackage{glossaries} %                                             
\usepackage{lastpage}       





%----------------------------------------------------------------------\|
%-----------------------------------FIN--- DE PACKETES------------------\
%\usepackage{listings}
%\lstset{
%  language=Scheme
%}

%--------------------------------/
%-------------------------------/
\usepackage[                 %   |
  headheight=15pt,  %            |
  letterpaper,  % Tipo de pag.   |
  left =1.5cm,  %  < 1 >         |
  right =1.5cm, %  < 1 >         | MARGENES DE LA PAGINA
  top =2cm,     %  < 1.5 >       |
  bottom =1.5cm %  < 1.5 >       |
]{geometry}     %                |
%-------------------------------\
%--------------------------------\

%----------------------------------------------------------------------/
%-------------------Encabezado y Pie de Pagina-----------------------/ |
%--------------------------------------------------------------------\ |
%\fancyhf{}           %                                                |

     %                                                |
%\pagestyle{fancy}

\fancypagestyle{firstpage}{  
  \fancyfoot[L]{\textsc{\textcolor{white}{\small {Fundamentos de Bases de Datos}}}}
  \fancyfoot[C]{}
  \fancyfoot[R]{\textcolor{white}{\thepage\ de \pageref*{LastPage}}} 
  \renewcommand{\footrulewidth}{1.5pt} %     | 
\xpretocmd\footrule{\color{white}}{}{\PatchFailed}
}

\fancypagestyle{fancy}{  

\fancyhead[L]{{\textcolor{white}{2024-1}}} %                         
\fancyhead[R]{\textcolor{white}{}}     % |
\fancyfoot[L]{\textsc{\textcolor{white}{\small {Fundamentos de Bases de Datos}}}}
  \fancyfoot[C]{}
  \fancyfoot[R]{\textcolor{white}{\thepage\ de \pageref*{LastPage}}} 
\renewcommand{\headrulewidth}{1pt} %
\xpretocmd\headrule{\color{white}}{}{\PatchFailed}
\renewcommand{\footrulewidth}{1.5pt} %     | 
\xpretocmd\footrule{\color{white}}{}{\PatchFailed}

}


%--------------------------------------------------------------------\ |
%-------------------Encabezado y Pie de Pagina-----------------------/ |
%------------------------------------------------------------FIN----/

%--------------------------------------------------------------------\ |
%------------------- LISTA DE COLORES -------------------------------/ |
\definecolor{ProcessBlue}{RGB}{0,176,240}
\definecolor{NavyBlue}{RGB}{0,110,184}
\definecolor{Cyan}{RGB}{0,174,239}
\definecolor{MidnightBlue}{RGB}{0,103,49}
\definecolor{ForestGreen}{RGB}{0,155,85}
\definecolor{Goldenrod}{RGB}{255,223,66}
\definecolor{YellowGreen}{RGB}{152,204,112}
\definecolor{Sepia}{RGB}{103,24,0}
\definecolor{Peach}{RGB}{247,150,90}
\definecolor{CarnationPink}{RGB}{242,130,180}
\definecolor{Fuchsia}{RGB}{140,54,140}
\definecolor{WildStrawberry}{RGB}{238,41,103}

\definecolor{Grass}{RGB}{41,238,53}
\definecolor{Meadow}{RGB}{6,243,67}
\definecolor{jellyfish}{RGB}{109,14,130}
\definecolor{rubber}{RGB}{229,27,232}
\definecolor{bullet}{RGB}{225,31,90}
\definecolor{midnight}{RGB}{31,90,225}
\definecolor{sun}{RGB}{241,152,7}
\definecolor{water}{RGB}{16,229,183}
%------------------- LISTA DE COLORES -------------------------------/ |
%----------------------------------------------------------------------/

\usepackage{tikz,times}
\usepackage{verbatim}
\usetikzlibrary{mindmap,trees,backgrounds}

\definecolor{color_mate}{RGB}{255,255,128}
\definecolor{color_plas}{RGB}{255,128,255}
\definecolor{color_text}{RGB}{128,255,255}
\definecolor{color_petr}{RGB}{255,192,192}
\definecolor{color_made}{RGB}{192,255,192}
\definecolor{color_meta}{RGB}{192,192,255}

\begin{document}%----------------------INICIO DOCUMENTO------------|
%------------------------------------------------------------------|
\pagecolor{black}
\color{white}

\thispagestyle{firstpage} % Aplicar estilo de primera página
\noindent
%%%%%%%%%%%%%%%%%%%%%%%%%%%%%%%%%%%%%%%%%%%%%%%%%%%%%%%%%%%%%%%%%%%%%%%%%%%%%%
\large\textbf{Facultad de Ciencias} \\
Fundamentos de bases de datos \hfill semestre: 2024$-$1 \\
\textsc{SILVA HUERTA MARCO}   \hfill     \\
11 de Octubre de 2023      \hfill \textbf{Tarea \#07}    \\
\noindent\rule{7.3in}{2.8pt}
%%%%%%%%%%%%%%%%%%%%%%%%%%%%%%%%%%%%%%%%%%%%%%%%%%%%%%%%%%%%%%%%%%%%%%%%%%%%%%

\begin{center}
    \Large{Modelo Relacional}
\end{center}

\begin{enumerate}
%%%%%%%%%%%%%%%%%%%%%%%%%%%
%%%%%% Pregunta 1 %%%%%%
%%%%%%%%%%%%%%%%%%%%%%%%%%%
    \item \textcolor{sun}{Completa el modelo relacional que se muestra a continuación, 
                            debes agregar claves primarias PK y las flechas correspondientes 
                            que representen a las claves foráneas FK:}
    
    \begin{itemize}
        \item \textbf{Cuenta} (\underline{nombreSucursal, numCta}, saldo)        
        \item \textbf{Sucursal} (\underline{nombreSucursal}, ciudad, activos)
        \item \textbf{Cliente} (\underline{nombreCliente}, calle, ciudad)
        \item \textbf{CtaCliente} (\underline{nombreCliente, numCta},  
        \begin{itemize}
            \item[] FK\: nombreCliente,
            \item[] numCta $\rightarrow$ Cuenta)
        \end{itemize}
        \item \textbf{Prestamo} (\underline{nombreSucursal, numPrestamo},   
        \begin{itemize}
            \item[] importe,
            \item[] FK\: nombreSucursal,
            \item[] numPrestamo $\rightarrow$ Prestamo)
        \end{itemize}        
        \item \textbf{Prestatario} (\underline{nombreCliente, numPrestamo},  
        \begin{itemize}
            \item[] FK\: nombreCliente,
            \item[] numPrestamo $\rightarrow$ Prestamo)
        \end{itemize}
    \end{itemize}
                            

%%%%%%%%%%%%%%%%%%%%%%%%%%%
%%%%%% Pregunta 2 %%%%%%
%%%%%%%%%%%%%%%%%%%%%%%%%%%
    \item \textcolor{sun}{Realiza las siguientes consultas por medio de Álgebra Relacional.}
    \begin{itemize}

        \item[(a)] Encontrar la información de todos los préstamos realizados en la sucursal llamada "Fuentes Brotantes"
                
        
        $\pi_{numPrestamo, importe}(\sigma_{nombreSucursal="Fuentes Brotantes"}(Prestamo))$

        
        \item[(b)] Determinar el nombre de los clientes que viven en Guanajuato.

        $(\sigma_{\text{ciudad} = \text{Guanajuato}} ( \text{Cliente} ))$

        
        \item[(c)] Nombre de los clientes del banco que tienen una cuenta, un préstamo o ambas cosas.

        $\pi_{\text{nombreCliente}}(\text{Cliente}) \cup \pi_{\text{nombreCliente}}(\text{CtaCliente}) \cup \pi_{\text{nombreCliente}}(\text{Prestatario})$

        
        \item[(d)] El listado de clientes que tienen abierta una cuenta pero no tienen ninguna de préstamo.

        $\pi_{\text{nombreCliente}}(\text{Cliente} - \text{CtaCliente})$

        
        \item[(e)] Todos los clientes que tienen un préstamo y una cuenta abierta.

        $\pi_{\text{nombreCliente}}(\text{CtaCliente} \cap \text{Prestatario})$

        
        \item[(f)] Nombre de los clientes que tienen un préstamo en la sucursal llamada Fuentes Brotantes.

        $\pi_{\text{nombreCliente}}(\sigma_{\text{nombreSucursal}=\text{Fuentes Brotantes}}(\text{Prestatario} \bowtie \text{Prestamo}))$
        
        \item[(g)] Nombre de los clientes que tienen un préstamo y el importe del mismo.

        $\pi_{\text{nombreCliente, importe}}(\text{Prestatario} \bowtie \text{Prestamo})$

        
        \item[(h)] Nombre de los clientes con préstamo mayor a cinco mil pesos.

        $\pi_{\text{nombreCliente}}(\sigma_{\text{importe}>5000}(\text{Prestamo} \bowtie \text{Prestatario}))$

        
        \item[(i)] Nombre de los clientes que tienen una cuenta con saldo menor a tres mil pesos y que no tienen préstamo.

        $\pi_{\text{nombreCliente}}(\sigma_{\text{saldo}<3000}(\text{CtaCliente}) - \pi_{\text{nombreCliente}}(\text{Prestatario}))$
        
    \end{itemize}
        

\end{enumerate}
\thispagestyle{fancy}    

    
%\newpage
%\thispagestyle{fancy} % Aplicar estilo de primera página
%\thispagestyle{firstpage} % Aplicar estilo de primera página
%\printbibliography[title={Bibliografia:}]%


%\printbibliography


%\newpage
%\thispagestyle{fancy} % Aplicar estilo de primera página
%\NoBgThispage % Quita el fondo de la pagina

%-----------------------------------------------------------------|
\end{document}%----------------------FIN DEL DOCUMENTO------------| 