%%%%%----------------------------------------
%%%%%--------------------------------------
%%%%%-------------------------------------
\section{\texttt{Naves}}
%%%%%-------------------------------------
%%%%%--------------------------------------
%%%%%----------------------------------------



%%%%%--------------------------------------
%%%%%-------------------------------------
\subsection{Lista de Requerimientos}
%%%%%-------------------------------------
%%%%%--------------------------------------

\subsubsection*{Tablas para la base de datos}

\begin{itemize}
    \item PASAJEROS
    \begin{itemize}
        \item Clave primaria: pasajeroID
        \item Nombre
        \item Apellido
        \item Edad
        \item Género
        \item Nacionalidad
        \item Clave foránea: naveID
    \end{itemize}
    \item NAVES
    \begin{itemize}
        \item Clave primaria: naveID
        \item Nombre
        \item Tipo
        \item Capacidad_Pasajeros
        \item VelocidadMax
        \item Clave foránea: rutaID
        \item Clave foránea: choferID
    \end{itemize}
    \item PLANETAS
    \begin{itemize}
        \item Clave primaria: planetaID
        \item Nombre
        \item DistanciaSol
        \item Gravedad
        \item Población
        \end{itemize}
    \item CHOFERES
    \begin{itemize}
        \item Clave primaria: choferID
        \item Nombre
        \item Edad
        \item Género
        \item Nacionalidad
        \item Clave foránea: ID_Nave
    \end{itemize}
    \item RUTAS
    \begin{itemize}
        \item Clave primaria: ID_Ruta
        \item Origen (ID_Planeta)
        \item Destino (ID_Planeta)
        \item Distancia
        \item Duración (tiempo estimado de viaje)
    \end{itemize}
\end{itemize}
\subsubsection*{Principales restricciones}
\begin{itemize}
    \item TABLAS
\end{itemize}

  
%%%%%--------------------------------------
%%%%%-------------------------------------
\subsection{Modelo conceptual}
%%%%%-------------------------------------
%%%%%--------------------------------------
\begin{center}
    MAPA
\end{center}


%%%%%--------------------------------------
%%%%%-------------------------------------
\subsection{Modelo relacional}
%%%%%-------------------------------------
%%%%%--------------------------------------
\begin{center}
  MAPA
\end{center}


%%%%%--------------------------------------
%%%%%-------------------------------------
\subsection{Script de creación}
%%%%%-------------------------------------
%%%%%--------------------------------------
\begin{lstlisting}[caption={Tablas para la BdDatos}, label={lst:sql_estadios}]
    -- Solo los titulos de las tablas para naves    
   
\end{lstlisting}

%%%%%--------------------------------------
%%%%%-------------------------------------
\subsection{Script de Insert}
%%%%%-------------------------------------
%%%%%--------------------------------------
\begin{itemize}
    \item[$\rightarrow$] Se deben generar 100 registros para cada tabla.
    \item[$\rightarrow$] Si para el buen funcionamiento de la base de datos se requieren más de 100 registros o
            menos de 100 registros en una tabla, se debe explicar claramente la razón, sólo en este caso
            sí se debe incluir un apartado en el reporte final.
\end{itemize}

Tablas:
\begin{itemize}
  \item TABLAS
\end{itemize}

%%%%%--------------------------------------
%%%%%-------------------------------------
\subsection{Funcionamiento Restricciones}
%%%%%-------------------------------------
%%%%%--------------------------------------

Evidencia del funcionamiento de al menos 4 restricciones de integridad referencial.

\subsubsection*{Restricción 01}

\begin{itemize}
    \item[$\rightarrow$] Tablas involucradas en la restricción: 
    \item[$\rightarrow$] FK de la tabla que referencia y PK de la tabla referenciada: 
    \item[$\rightarrow$] Justificación del trigger de integridad referencial elegido: 
    \item[$\rightarrow$] Instrucción UPDATE o DELETE que permita evidenciar que la restricción está funcionando.
        \begin{verbatim}
    
        \end{verbatim}

    \item[$\rightarrow$] Captura de pantalla con el resultado de la instrucción que muestre que la restricción está funcionando.
    
        \begin{center}
            CAPTURA
        \end{center}

\end{itemize}

\subsubsection*{Restricción 02}

\begin{itemize}
    \item[$\rightarrow$] Tablas involucradas en la restricción: 
    \item[$\rightarrow$] FK de la tabla que referencia y PK de la tabla referenciada: 
    \item[$\rightarrow$] Justificación del trigger de integridad referencial elegido: 
    \item[$\rightarrow$] Instrucción UPDATE o DELETE que permita evidenciar que la restricción está funcionando.
        \begin{verbatim}
    
        \end{verbatim}

    \item[$\rightarrow$] Captura de pantalla con el resultado de la instrucción que muestre que la restricción está funcionando.
    
        \begin{center}
            CAPTURA
        \end{center}

\end{itemize}

\subsubsection*{Restricción 03}

\begin{itemize}
    \item[$\rightarrow$] Tablas involucradas en la restricción: 
    \item[$\rightarrow$] FK de la tabla que referencia y PK de la tabla referenciada: 
    \item[$\rightarrow$] Justificación del trigger de integridad referencial elegido: 
    \item[$\rightarrow$] Instrucción UPDATE o DELETE que permita evidenciar que la restricción está funcionando.
        \begin{verbatim}
    
        \end{verbatim}

    \item[$\rightarrow$] Captura de pantalla con el resultado de la instrucción que muestre que la restricción está funcionando.
    
        \begin{center}
            CAPTURA
        \end{center}

\end{itemize}

\subsubsection*{Restricción 04}

\begin{itemize}
    \item[$\rightarrow$] Tablas involucradas en la restricción: 
    \item[$\rightarrow$] FK de la tabla que referencia y PK de la tabla referenciada: 
    \item[$\rightarrow$] Justificación del trigger de integridad referencial elegido: 
    \item[$\rightarrow$] Instrucción UPDATE o DELETE que permita evidenciar que la restricción está funcionando.
        \begin{verbatim}
    
        \end{verbatim}

    \item[$\rightarrow$] Captura de pantalla con el resultado de la instrucción que muestre que la restricción está funcionando.
    
        \begin{center}
            CAPTURA
        \end{center}

\end{itemize}


%%%%%--------------------------------------
%%%%%-------------------------------------
\subsection{Funcionamiento Restricciones check}
%%%%%-------------------------------------
%%%%%--------------------------------------

Evidencia del funcionamiento de al menos 3 restricciones check para “atributos” de varias
tablas.

\subsubsection*{Restricción 01}

\begin{itemize} 
  \item Tabla elegida: 
  \item Atributo elegido: 
  \item Descripción: 
  \item Instrucción para la creación de la restricción: 
  \item 
    \begin{verbatim} 
    
    \end{verbatim} 

  \item Instrucción que permita evidenciar que la restricción esta funcionando: 
  
    \begin{verbatim} 
    
    \end{verbatim} 

  \item Captura de pantalla 
  
    \begin{center} 
      CAPTURA
    \end{center} 

\end{itemize}


\subsubsection*{Restricción 02}

\begin{itemize} 
  \item Tabla elegida: 
  \item Atributo elegido: 
  \item Descripción: 
  \item Instrucción para la creación de la restricción: 
  \item 
    \begin{verbatim} 
    
    \end{verbatim} 

  \item Instrucción que permita evidenciar que la restricción esta funcionando: 
  
    \begin{verbatim} 
    
    \end{verbatim} 

  \item Captura de pantalla 
  
    \begin{center} 
      CAPTURA
    \end{center} 

\end{itemize}


\subsubsection*{Restricción 03}

\begin{itemize} 
  \item Tabla elegida: 
  \item Atributo elegido: 
  \item Descripción: 
  \item Instrucción para la creación de la restricción: 
  \item 
    \begin{verbatim} 
    
    \end{verbatim} 

  \item Instrucción que permita evidenciar que la restricción esta funcionando: 
  
    \begin{verbatim} 
    
    \end{verbatim} 

  \item Captura de pantalla 
  
    \begin{center} 
      CAPTURA
    \end{center} 

\end{itemize}


%%%%%--------------------------------------
%%%%%-------------------------------------
\subsection{Creación de dominios personalizados}
%%%%%-------------------------------------
%%%%%--------------------------------------

Evidencia de la creación de al menos tres dominios personalizados. Se deben utilizar restricciones check en la creación de los tres dominios.

\subsubsection*{Dominio 01}

\begin{itemize}
    \item Tabla elegida: 
    \item Atributo elegido: 
    \item Descripción: 
    \item Instrucción para la creación del dominio personalizado.
    \begin{lstlisting}[caption={Tablas para la BdDatos}, label={lst:sql_estadios}]
    
    \end{lstlisting}
\end{itemize}


\subsubsection*{Dominio 02}

\begin{itemize}
    \item Tabla elegida: 
    \item Atributo elegido: 
    \item Descripción: 
    \item Instrucción para la creación del dominio personalizado.
    \begin{lstlisting}[caption={Tablas para la BdDatos}, label={lst:sql_estadios}]
    
    \end{lstlisting}
\end{itemize}


\subsubsection*{Dominio 03}

\begin{itemize}
    \item Tabla elegida: 
    \item Atributo elegido: 
    \item Descripción: 
    \item Instrucción para la creación del dominio personalizado.
    \begin{lstlisting}[caption={Tablas para la BdDatos}, label={lst:sql_estadios}]
    
    \end{lstlisting}
\end{itemize}



%%%%%--------------------------------------
%%%%%-------------------------------------
\subsection{Restricciones para tuplas}
%%%%%-------------------------------------
%%%%%--------------------------------------

Evidencia del funcionamiento de al menos 2 restricciones para tuplas en diferentes tablas (Unidad 8 Integridad, tema Specifying Constraints on Tuples Using CHECK)


\subsubsection*{Dominio 01}

\begin{itemize}
    \item Tabla elegida: 
    \item Breve descripción de la restricción: 
    \item Instrucción para la creación de la restricción:
        \begin{verbatim}
        
        \end{verbatim}

    \item Instrucción Insert o Update que permita evidenciar que la restricción esta funcionando:
    
        \begin{verbatim}
        
        \end{verbatim}  
\end{itemize}


\subsubsection*{Dominio 02}

\begin{itemize}
    \item Tabla elegida: 
    \item Breve descripción de la restricción: 
    \item Instrucción para la creación de la restricción:
        \begin{verbatim}
        
        \end{verbatim}

    \item Instrucción Insert o Update que permita evidenciar que la restricción esta funcionando:
    
        \begin{verbatim}
        
        \end{verbatim}  
\end{itemize}


%%%%%--------------------------------------
%%%%%-------------------------------------
\subsection{Consultas}
%%%%%-------------------------------------
%%%%%--------------------------------------

Plantea 3 consultas que consideres relevantes para la base de datos propuesta. Para cada consulta planteada, incluir en el reporte los siguientes incisos:


\subsubsection*{Consulta 01}

\begin{itemize}
    \item Redacción clara de la consulta: 
    \item Código en lenguaje SQL de la consulta.
    
    \begin{lstlisting}[caption={Tablas para la BdDatos}, label={lst:sql_estadios}]
    
    \end{lstlisting}    

    \begin{center}
      CAPTURA
    \end{center}
    
\end{itemize}


\subsubsection*{Consulta 02}

\begin{itemize}
    \item Redacción clara de la consulta: 
    \item Código en lenguaje SQL de la consulta.
    
    \begin{lstlisting}[caption={Tablas para la BdDatos}, label={lst:sql_estadios}]
    
    \end{lstlisting}    

    \begin{center}
      CAPTURA
    \end{center}
    
\end{itemize}


\subsubsection*{Consulta 03}

\begin{itemize}
    \item Redacción clara de la consulta: 
    \item Código en lenguaje SQL de la consulta.
    
    \begin{lstlisting}[caption={Tablas para la BdDatos}, label={lst:sql_estadios}]
    
    \end{lstlisting}    

    \begin{center}
      CAPTURA
    \end{center}
    
\end{itemize}


%%%%%--------------------------------------
%%%%%-------------------------------------
\subsection{Vistas}
%%%%%-------------------------------------
%%%%%--------------------------------------

Plantea 3 vistas que consideres relevantes para la base de datos propuesta. Para cada vista planteada, incluir en el reporte los siguientes incisos:

\subsection*{Vista 01}

\begin{itemize}
    \item Redacción clara de la vista planteada: 
    \item Código en lenguaje SQL que permita crear la vista solicitada.
    
    \begin{lstlisting}[caption={Tablas para la BdDatos}, label={lst:sql_estadios}]
      
    \end{lstlisting}    
    
    \item USO:
    
    \begin{lstlisting}[caption={Tablas para la BdDatos}, label={lst:sql_estadios}]
      
    \end{lstlisting}    

    \begin{center}
      CAPTURA
    \end{center}
\end{itemize}


\subsection*{Vista 02}

\begin{itemize}
    \item Redacción clara de la vista planteada: 
    \item Código en lenguaje SQL que permita crear la vista solicitada.
    
    \begin{lstlisting}[caption={Tablas para la BdDatos}, label={lst:sql_estadios}]
      
    \end{lstlisting}    
    
    \item USO:
    
    \begin{lstlisting}[caption={Tablas para la BdDatos}, label={lst:sql_estadios}]
      
    \end{lstlisting}    

    \begin{center}
      CAPTURA
    \end{center}
\end{itemize}


\subsection*{Vista 03}

\begin{itemize}
    \item Redacción clara de la vista planteada: 
    \item Código en lenguaje SQL que permita crear la vista solicitada.
    
    \begin{lstlisting}[caption={Tablas para la BdDatos}, label={lst:sql_estadios}]
      
    \end{lstlisting}    
    
    \item USO:
    
    \begin{lstlisting}[caption={Tablas para la BdDatos}, label={lst:sql_estadios}]
      
    \end{lstlisting}    

    \begin{center}
      CAPTURA
    \end{center}
\end{itemize}