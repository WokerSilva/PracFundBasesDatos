%%%%%----------------------------------------
%%%%%--------------------------------------
%%%%%-------------------------------------
\section{\texttt{Naves}}
%%%%%-------------------------------------
%%%%%--------------------------------------
%%%%%----------------------------------------

\subsection{Requerimientos}

La empresa Galactic Travel Solutions ha solicitado el diseño de una base de datos para gestionar 
los viajes galácticos en el año 4000. El objetivo es administrar eficientemente la información relacionada 
con pasajeros, naves, planetas, rutas y choferes para ofrecer servicios de transporte 
interplanetario de manera segura y organizada.

\begin{itemize}
  \item Cada pasajero debe estar asociado a una única nave, y cada nave debe tener asignada una ruta específica.
  \item Los planetas deben tener información única y estar relacionados con las rutas para garantizar la coherencia de los viajes.
  \item Las relaciones entre las tablas deben ser mantenidas para asegurar la integridad referencial. Por ejemplo, un chofer debe estar asignado a una nave existente.
  \item Los identificadores primarios y foráneos deben seguir una convención clara y única para facilitar la comprensión y mantenimiento de la base de datos.
  \item Se deben establecer restricciones de integridad para prevenir operaciones que puedan comprometer la consistencia de los datos, como eliminar un planeta asociado a una ruta activa.
\end{itemize}

%%%%%--------------------------------------
%%%%%-------------------------------------
\subsection{Modelo conceptual}
%%%%%-------------------------------------
%%%%%--------------------------------------
\begin{center}
    \includegraphics[scale = .6]{IMA/nave/BD-NAVE-MO-CONCEPTUAL.png}
\end{center}


%%%%%--------------------------------------
%%%%%-------------------------------------
\subsection{Modelo relacional}
%%%%%-------------------------------------
%%%%%--------------------------------------
\begin{center}
  \includegraphics[scale = .7]{IMA/nave/BD-NAVE-MO-RELACIONAL.png}
\end{center}


%%%%%--------------------------------------
%%%%%-------------------------------------
\subsection{Script de creación}
%%%%%-------------------------------------
%%%%%--------------------------------------
\begin{lstlisting}[caption={Tablas para la BdDatos}, label={lst:sql_estadios}]
    -- Solo los titulos de las tablas para naves    
    CREATE TABLE PLANETAS
    (
      
    );
    
    CREATE TABLE RUTAS
    (

    );
    
    
    CREATE TABLE NAVES
    (
      
    );
    
    CREATE TABLE CHOFERES
    (
      
    );
    
    CREATE TABLE PASAJEROS
    (

    );
\end{lstlisting}

%%%%%--------------------------------------
%%%%%-------------------------------------
\subsection{Script de Insert}
%%%%%-------------------------------------
%%%%%--------------------------------------
\begin{itemize}
    \item[$\rightarrow$] Se deben generar 100 registros para cada tabla.
    \item[$\rightarrow$] Si para el buen funcionamiento de la base de datos se requieren más de 100 registros o
            menos de 100 registros en una tabla, se debe explicar claramente la razón, sólo en este caso
            sí se debe incluir un apartado en el reporte final.
\end{itemize}

Tablas:
\begin{itemize}
  \item PLANETAS: no es posible porque nos concentramos en los 8 planetas del sistema solar
  \item RUTAS: no es posible porque se toman las posibles rutas entre los 8 planetas, 28
  \item NAVES: no es posible porque se toman 28 naves para las 28 rutas
  \item CHOFERES: no es posible porque se toman 28 choferes para las 28 rutas
  
  Es importante resaltar que para respetar las secuencias y no tener problemas con los ID se tuvo que realizar así la tabla

  \item PASAJEROS: Si es posible llegar a los 100 registros.
\end{itemize}

%%%%%--------------------------------------
%%%%%-------------------------------------
\subsection{Funcionamiento Restricciones}
%%%%%-------------------------------------
%%%%%--------------------------------------

Evidencia del funcionamiento de al menos 4 restricciones de integridad referencial.

\subsubsection*{Restricción 01}

\begin{itemize}
  \item[$\rightarrow$] Tablas involucradas en la restricción: NAVES y RUTAS
  \item[$\rightarrow$] FK de la tabla que referencia y PK de la tabla referenciada: rutaID en NAVES (FK) hace referencia a rutaID en RUTAS (PK)
  \item[$\rightarrow$] Justificación del trigger de integridad referencial elegido: Esta restricción asegura que cada rutaID en la tabla NAVES corresponda a un rutaID válido en la tabla RUTAS. Esto es crucial para mantener la coherencia de los datos, ya que no tendría sentido tener una nave asignada a una ruta que no existe.
  \item[$\rightarrow$] Instrucción UPDATE o DELETE que permita evidenciar que la restricción está funcionando.
  \begin{verbatim}  
  DELETE FROM RUTAS WHERE rutaID = 'R1';  
  \end{verbatim}
  \item[$\rightarrow$] Captura de pantalla con el resultado de la instrucción que muestre que la restricción está funcionando.    
        \begin{center}
            \includegraphics[scale = .4]{IMA/nave/6-01.png}
        \end{center}

\end{itemize}

\subsubsection*{Restricción 02}

\begin{itemize}
  \item[$\rightarrow$] Tablas involucradas en la restricción: CHOFERES y NAVES
  \item[$\rightarrow$] FK de la tabla que referencia y PK de la tabla referenciada: naveID en CHOFERES (FK) hace referencia a naveID en NAVES (PK)
  \item[$\rightarrow$] Justificación del trigger de integridad referencial elegido: Esta restricción asegura que cada naveID en la tabla CHOFERES corresponda a un naveID válido en la tabla NAVES. Esto es crucial para mantener la coherencia de los datos, ya que no tendría sentido tener un chofer asignado a una nave que no existe.
  \item[$\rightarrow$] Instrucción UPDATE o DELETE que permita evidenciar que la restricción está funcionando.
  \begin{verbatim}
  DELETE FROM NAVES WHERE naveID = 'N19RR';
  \end{verbatim}
  \item[$\rightarrow$] Captura de pantalla con el resultado de la instrucción que muestre que la restricción está funcionando.    
        \begin{center}
          \includegraphics[scale = .4]{IMA/nave/6-02.png}
        \end{center}

\end{itemize}

\subsubsection*{Restricción 03}

\begin{itemize}
  \item[$\rightarrow$] Tablas involucradas en la restricción: PASAJEROS y NAVES
  \item[$\rightarrow$] FK de la tabla que referencia y PK de la tabla referenciada: naveID en PASAJEROS (FK) hace referencia a naveID en NAVES (PK)
  \item[$\rightarrow$] Justificación del trigger de integridad referencial elegido: Esta restricción asegura que cada naveID en la tabla PASAJEROS corresponda a un naveID válido en la tabla NAVES. Esto es crucial para mantener la coherencia de los datos, ya que no tendría sentido tener un pasajero asignado a una nave que no existe.
  \item[$\rightarrow$] Instrucción UPDATE o DELETE que permita evidenciar que la restricción está funcionando.
  \begin{verbatim}
  DELETE FROM NAVES WHERE naveID = 'N7SR';
  \end{verbatim}
  \item[$\rightarrow$] Captura de pantalla con el resultado de la instrucción que muestre que la restricción está funcionando.    
        \begin{center}
          \includegraphics[scale = .4]{IMA/nave/6-03.png}
        \end{center}

\end{itemize}

\subsubsection*{Restricción 04}

\begin{itemize}
  \item[$\rightarrow$] Tablas involucradas en la restricción: RUTAS y PLANETAS
  \item[$\rightarrow$] FK de la tabla que referencia y PK de la tabla referenciada: origen y destino en RUTAS (FK) hacen referencia a planetaID en PLANETAS (PK)
  \item[$\rightarrow$] Justificación del trigger de integridad referencial elegido: Esta restricción asegura que cada origen y destino en la tabla RUTAS corresponda a un planetaID válido en la tabla PLANETAS. Esto es crucial para mantener la coherencia de los datos, ya que no tendría sentido tener una ruta que comienza o termina en un planeta que no existe.
  \item[$\rightarrow$] Instrucción UPDATE o DELETE que permita evidenciar que la restricción está funcionando.
  \begin{verbatim}  
  DELETE FROM PLANETAS WHERE planetaID = 'Tierra';  
  \end{verbatim}
  \item[$\rightarrow$] Captura de pantalla con el resultado de la instrucción que muestre que la restricción está funcionando.
    
        \begin{center}
          \includegraphics[scale = .4]{IMA/nave/6-04.png}
        \end{center}

\end{itemize}


%%%%%--------------------------------------
%%%%%-------------------------------------
\subsection{Funcionamiento Restricciones check}
%%%%%-------------------------------------
%%%%%--------------------------------------

Evidencia del funcionamiento de al menos 3 restricciones check para “atributos” de varias
tablas.

\subsubsection*{Restricción 01}

\begin{itemize} 
  \item Tabla elegida: PLANETAS 
  \item Atributo elegido: distanciaSol 
  \item Descripción: Esta restricción asegura que la distancia de un planeta al sol sea siempre positiva. 
  \item Instrucción para la creación de la restricción: 
    \begin{verbatim} 
    ALTER TABLE PLANETAS 
    ADD CONSTRAINT distanciaSol CHECK (distanciaSol > 0); 
    \end{verbatim} 
  \item Instrucción que permita evidenciar que la restricción está funcionando: 
    \begin{verbatim} 
    INSERT INTO PLANETAS (planetaID, estacion, distanciaSol, 
                          gravedad, poblacion) 
      VALUES ('Pluton', 'Deep', -100, 9.8, 1000000);
    \end{verbatim} 
  \item Captura de pantalla 
  \begin{center}
    \includegraphics[scale = .5]{IMA/nave/7-01.png}
  \end{center} 
\end{itemize}

\subsubsection*{Restricción 02}

\begin{itemize} 
  \item Tabla elegida: NAVES 
  \item Atributo elegido: capacidad 
  \item Descripción: Esta restricción asegura que la capacidad de una nave sea siempre positiva. 
  \item Instrucción para la creación de la restricción: 
    \begin{verbatim}
    ALTER TABLE NAVES 
    ADD CONSTRAINT capacidad CHECK (capacidad > 0); 
    \end{verbatim} 
  \item Instrucción que permita evidenciar que la restricción está funcionando: 
    \begin{verbatim} 
    INSERT INTO NAVES (naveID, modelo, capacidad, velocidadMax, rutaID) 
    VALUES ('N002', 'Modelo1', -50, 1000, 'R001'); 
    \end{verbatim} 
    
  \item Captura de pantalla 
    \begin{center} 
      \includegraphics[scale = .5]{IMA/nave/7-02.png}
    \end{center} 
\end{itemize}


\subsubsection*{Restricción 03}

\begin{itemize} 
  \item Tabla elegida: CHOFERES 
  \item Atributo elegido: edad 
  \item Descripción: Esta restricción asegura que la edad de un chofer sea siempre positiva y menor que 100. 
  \item Instrucción para la creación de la restricción: 
    \begin{verbatim} 
    ALTER TABLE CHOFERES
    ADD CONSTRAINT edad CHECK (edad > 0 AND edad < 100); 
    \end{verbatim} 
  \item Instrucción que permita evidenciar que la restricción está funcionando: 
    \begin{verbatim}
      INSERT INTO CHOFERES (choferID, nombre, edad, genero, nacionalidad, naveID) 
      VALUES ('C002', 'Chofer1', -30, 'M', 'Noruega', 'N001');
    \end{verbatim} 
  \item Captura de pantalla 
    \begin{center}
      \includegraphics[scale = .5]{IMA/nave/7-03.png}
    \end{center} 
\end{itemize}


%%%%%--------------------------------------
%%%%%-------------------------------------
\subsection{Creación de dominios personalizados}
%%%%%-------------------------------------
%%%%%--------------------------------------

Evidencia de la creación de al menos tres dominios personalizados. Se deben utilizar restricciones check en la creación de los tres dominios.

\subsubsection*{Dominio 01}

\begin{itemize} 
  \item Tabla elegida: PLANETAS 
  \item Atributo elegido: distanciaSol 
  \item Descripción: Este dominio personalizado asegura que la distancia de un planeta al sol sea siempre positiva y menor que 1000000 (un millón de unidades de distancia). 
  \item Instrucción para la creación del dominio personalizado. 
  \begin{lstlisting}[caption={Dominio 01}, label={lst:sql_estadios}] 
    distanciaSol INT CHECK (distanciaSol > 0 AND distanciaSol < 1000000),
  \end{lstlisting} 
\end{itemize}

\subsubsection*{Dominio 02}

\begin{itemize} 
  \item Tabla elegida: NAVES 
  \item Atributo elegido: capacidad 
  \item Descripción: Este dominio personalizado asegura que la capacidad de una nave sea siempre positiva y menor que 1000. 
  \item Instrucción para la creación del dominio personalizado. 
    \begin{lstlisting}[caption={Dominio 02}, label={lst:sql_estadios}] 
    CREATE DOMAIN capacidad AS INT CHECK (VALUE > 0 AND VALUE < 300); 
    \end{lstlisting} 
\end{itemize}


\subsubsection*{Dominio 03}

\begin{itemize} 
  \item Tabla elegida: CHOFERES 
  \item Atributo elegido: edad 
  \item Descripción: Este dominio personalizado asegura que la edad de un chofer sea siempre positiva y menor que 100. 
  \item Instrucción para la creación del dominio personalizado. 
    \begin{lstlisting}[caption={Dominio 03}, label={lst:sql_estadios}] 
      CREATE DOMAIN EdadChofer AS INT CHECK (VALUE > 0 AND VALUE < 100); 
    \end{lstlisting} 
\end{itemize}



%%%%%--------------------------------------
%%%%%-------------------------------------
\subsection{Restricciones para tuplas}
%%%%%-------------------------------------
%%%%%--------------------------------------

Evidencia del funcionamiento de al menos 2 restricciones para tuplas en diferentes tablas (Unidad 8 Integridad, tema Specifying Constraints on Tuples Using CHECK)


\subsubsection*{Restricción 01}

\begin{itemize} 
  \item Tabla elegida: NAVES 
  \item Breve descripción de la restricción: Esta restricción asegura que la capacidad de una nave siempre sea menor que su velocidadMax. 
  \item Instrucción para la creación de la restricción: 
    \begin{verbatim} 
      ALTER TABLE NAVES 
      ADD CONSTRAINT capacidad_velocidad CHECK (capacidad < velocidadMax); 
    \end{verbatim} 
  \item Instrucción Insert o Update que permita evidenciar que la restricción está funcionando: 
    \begin{verbatim} 
      INSERT INTO NAVES (naveID, modelo, capacidad, velocidadMax, rutaID) 
      VALUES ('N003', 'Modelo2', 2000, 1000, 'R001'); 
    \end{verbatim}
    \begin{center}
      \includegraphics[scale = .5]{IMA/nave/tu-01.png}
    \end{center}
\end{itemize}


\subsubsection*{Restricción 02}

\begin{itemize} 
  \item Tabla elegida: RUTAS 
  \item Breve descripción de la restricción: Esta restricción asegura que el origen y el destino de una ruta sean diferentes. 
  \item Instrucción para la creación de la restricción: 
    \begin{verbatim} 
      ALTER TABLE RUTAS 
      ADD CONSTRAINT origen_destino CHECK (origen <> destino); 
    \end{verbatim} 
  \item Instrucción Insert o Update que permita evidenciar que la restricción está funcionando: 
    \begin{verbatim}
    INSERT INTO RUTAS (rutaID, origen, destino, distancia, duracion) 
    VALUES ('R003', 'P001', 'P001', 1000, 10);
    \end{verbatim}
    \begin{center}
      \includegraphics[scale = .5]{IMA/nave/tu-02.png}
    \end{center}
\end{itemize}

%%%%%--------------------------------------
%%%%%-------------------------------------
\subsection{Consultas}
%%%%%-------------------------------------
%%%%%--------------------------------------

Plantea 3 consultas que consideres relevantes para la base de datos propuesta. Para cada consulta planteada, incluir en el reporte los siguientes incisos:


\subsubsection*{Consulta 01}

\begin{itemize} 
  \item Redacción clara de la consulta: Obtener la lista de todas las naves y sus respectivas rutas. 
  \item Código en lenguaje SQL de la consulta.

  \begin{lstlisting}[caption={Consulta de naves y rutas}, label={lst:sql_estadios}]
  SELECT N.naveID, N.modelo, R.rutaID, R.origen, R.destino
  FROM NAVES N
  INNER JOIN RUTAS R ON N.rutaID = R.rutaID;
  \end{lstlisting}    

    \begin{center}
      \includegraphics[scale = .5]{IMA/nave/con1.png}
    \end{center}
    
\end{itemize}


\subsubsection*{Consulta 02}

\begin{itemize} 
  \item Redacción clara de la consulta: Obtener la lista de todos los choferes y las naves que están manejando. 
  \item Código en lenguaje SQL de la consulta.

  \begin{lstlisting}[caption={Consulta de choferes y naves}, label={lst:sql_estadios}]
  SELECT C.choferID, C.nombre, N.naveID, N.modelo
  FROM CHOFERES C
  INNER JOIN NAVES N ON C.naveID = N.naveID;
  \end{lstlisting}    

    \begin{center}
      \includegraphics[scale = .5]{IMA/nave/con2.png}
    \end{center}
    
\end{itemize}


\subsubsection*{Consulta 03}

\begin{itemize}
    \item Redacción clara de la consulta: Obtener la lista de todos los pasajeros y las naves en las que están viajando.
    \item Código en lenguaje SQL de la consulta.
    
    \begin{lstlisting}[caption={Tablas para la BdDatos}, label={lst:sql_estadios}]
    SELECT PASAJEROS.nombre, PASAJEROS.apellido, NAVES.modelo
    FROM PASAJEROS
    INNER JOIN NAVES ON PASAJEROS.naveID = NAVES.naveID;
    \end{lstlisting}    

    \begin{center}
      \includegraphics[scale = .5]{IMA/nave/con3.png}
    \end{center}
    
\end{itemize}


%%%%%--------------------------------------
%%%%%-------------------------------------
\subsection{Vistas}
%%%%%-------------------------------------
%%%%%--------------------------------------

Plantea 3 vistas que consideres relevantes para la base de datos propuesta. Para cada vista planteada, incluir en el reporte los siguientes incisos:

\subsection*{Vista 01}

\begin{itemize}
    \item Redacción clara de la vista planteada: Esta vista mostrará todas las naves junto con sus rutas respectivas.
    \item Código en lenguaje SQL que permita crear la vista solicitada.
    
    \begin{lstlisting}[caption={Tablas para la BdDatos}, label={lst:sql_estadios}]
    CREATE VIEW Naves_Rutas AS
    SELECT NAVES.naveID, NAVES.modelo, RUTAS.origen, RUTAS.destino
    FROM NAVES
    INNER JOIN RUTAS ON NAVES.rutaID = RUTAS.rutaID;
    \end{lstlisting}    
    
    \item USO:
    
    \begin{lstlisting}[caption={Tablas para la BdDatos}, label={lst:sql_estadios}]
    SELECT * FROM Naves_Rutas;
    \end{lstlisting}    

    \begin{center}
      \includegraphics[scale = .5]{IMA/nave/V1.png}
    \end{center}
\end{itemize}


\subsection*{Vista 02}

\begin{itemize}
    \item Redacción clara de la vista planteada: Esta vista mostrará todos los choferes junto con las naves que están manejando.
    \item Código en lenguaje SQL que permita crear la vista solicitada.
    
    \begin{lstlisting}[caption={Tablas para la BdDatos}, label={lst:sql_estadios}]
      CREATE VIEW Choferes_Naves AS
      SELECT CHOFERES.nombre, NAVES.modelo
      FROM CHOFERES
      INNER JOIN NAVES ON CHOFERES.naveID = NAVES.naveID;      
    \end{lstlisting}    
    
    \item USO:
    
    \begin{lstlisting}[caption={Tablas para la BdDatos}, label={lst:sql_estadios}]
      SELECT * FROM Choferes_Naves;      
    \end{lstlisting}    

    \begin{center}
      \includegraphics[scale = .5]{IMA/nave/V2.png}
    \end{center}
\end{itemize}


\subsection*{Vista 03}

\begin{itemize}
    \item Redacción clara de la vista planteada: Esta vista mostrará todos los pasajeros junto con las naves en las que están viajando.
    \item Código en lenguaje SQL que permita crear la vista solicitada.
    
    \begin{lstlisting}[caption={Tablas para la BdDatos}, label={lst:sql_estadios}]
      CREATE VIEW Pasajeros_Naves AS
      SELECT PASAJEROS.nombre, PASAJEROS.apellido, NAVES.modelo
      FROM PASAJEROS
      INNER JOIN NAVES ON PASAJEROS.naveID = NAVES.naveID;      
    \end{lstlisting}    
    
    \item USO:
    
    \begin{lstlisting}[caption={Tablas para la BdDatos}, label={lst:sql_estadios}]
      SELECT * FROM Pasajeros_Naves;
    \end{lstlisting}    

    \begin{center}
      \includegraphics[scale = .5]{IMA/nave/V3.png}
    \end{center}
\end{itemize}