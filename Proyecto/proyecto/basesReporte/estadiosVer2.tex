%%%%%----------------------------------------
%%%%%--------------------------------------
%%%%%-------------------------------------
\section{\texttt{Estadios}}
%%%%%-------------------------------------
%%%%%--------------------------------------
%%%%%----------------------------------------


La FIFA nos ha encargado el desarrollo de un sistema para la administración de la venta de boletos
en estadios de fútbol para los partidos del Mundial de 2026. El sistema deberá cubrir los cuatro
estadios de fútbol de México que albergarán partidos del torneo: el Estadio Azteca,
el Estadio Corregidora, el Estadio Hidalgo y el Estadio Jalisco.


\subsection{Lista de Requerimientos}

\subsubsection*{El sistema deberá cumplir con los siguientes requisitos}

\begin{itemize}
    \item Permitir la venta de boletos por internet y en taquillas.
    \item Permitir la selección de asientos por parte de los clientes.
    \item Permitir el pago de los boletos con tarjeta de crédito, débito o efectivo.
    \item Generar reportes de ventas.
\end{itemize}

\subsubsection*{La base de datos deberá tener las siguientes tablas:}

\begin{itemize}
    \item \textbf{Boletos:}
        \begin{itemize}
            \item El número de boleto debe ser único.
            \item La fecha del partido debe ser una fecha válida.
            \item El estadio debe ser uno de los cuatro estadios de México.
            \item La sección debe ser una sección válida del estadio.
            \item El asiento debe ser un asiento válido de la sección.
            \item El precio debe ser un número positivo.
            \item El estado de venta debe ser Vendido o Disponible.
        \end{itemize}

    \item \textbf{Clientes:}
        \begin{itemize}
            \item El nombre del cliente debe ser una cadena no vacía.
            \item La dirección del cliente debe ser una cadena no vacía.
            \item El teléfono del cliente debe ser un número de teléfono válido.
            \item El correo electrónico del cliente debe ser una dirección de correo electrónico válida.
            \item La tarjeta de crédito debe ser un número de tarjeta de crédito válido.
        \end{itemize}

    \item \textbf{Estadios:}
        \begin{itemize}
            \item La capacidad del estadio debe ser un número positivo que represente la cantidad máxima de espectadores permitidos.
            \item La ubicación debe ser una cadena que describa la ciudad o lugar donde se encuentra el estadio.
            \item Debe haber al menos una sección asociada a cada estadio.
            \item El nombre del estadio debe ser único.
        \end{itemize}
    \item \textbf{Secciones:}
        \begin{itemize}
            \item El nombre de la sección debe ser único dentro de un estadio.
            \item La capacidad de la sección debe ser un número positivo que represente la cantidad máxima de espectadores que puede albergar la sección.
        \end{itemize}    

    \item \textbf{Transacciones:} 
        \begin{itemize}
            \item El número de transacción debe ser único.
            \item La fecha de la transacción debe ser una fecha válida.
            \item El cliente asociado a la transacción debe existir en la tabla Clientes.
            \item El boleto asociado a la transacción debe existir en la tabla Boletos.
            \item El precio en la transacción debe ser igual al precio del boleto multiplicado por la cantidad de boletos en la transacción.
        \end{itemize}

    \item \textbf{Equipos:} 
        \begin{itemize}
            \item El nombre del equipo debe ser único.
            \item El país del equipo debe ser una cadena no vacía que represente el país al que pertenece el equipo.
            \item El escudo y el color de uniforme deben ser cadenas que describan la imagen del escudo y el color de uniforme del equipo, respectivamente.
        \end{itemize}    
    
    \item \textbf{Partido:} 
    \begin{itemize}
        \item El id del partido debe ser único.
        \item La fecha y hora del partido deben ser válidas y estar en el futuro.
        \item Debe haber al menos dos equipos participando en cada partido.
        \item El estadio asociado al partido debe existir en la tabla Estadios.
    \end{itemize}
        
    \item \textbf{EquiposPartido:} 
    \begin{itemize}
        \item El id del equipo en el partido debe hacer referencia a un equipo existente en la tabla Equipos.
        \item El id del partido en el equipo partido debe hacer referencia a un partido existente en la tabla Partido.
        \item La combinación única de id del equipo y id del partido debe asegurar que un equipo no participe más de una vez en el mismo partido.
    \end{itemize}
\end{itemize}

\subsubsection*{Restricciones de los datos}

\begin{itemize}
    \item El número de boleto debe ser único.
    \item La fecha del partido debe ser una fecha válida.
    \item El estadio debe ser uno de los cuatro estadios de México.
    \item La sección debe ser una sección válida del estadio.
    \item El asiento debe ser un asiento válido de la sección.
    \item El precio debe ser un número positivo.
    \item El estado de venta debe ser Vendido o Disponible.
    \item El nombre del cliente debe ser una cadena no vacía.
    \item La dirección del cliente debe ser una cadena no vacía.
    \item El teléfono del cliente debe ser un número de teléfono válido.
    \item El correo electrónico del cliente debe ser una dirección de correo electrónico válida.
    \item El número de tarjeta de crédito debe ser un número de tarjeta de crédito válido.
    \item La fecha de la transacción debe ser una fecha válida.
    \item El equipo debe ser uno de los equipos participantes en el Mundial.
    \item La capacidad del estadio debe ser un número positivo que represente la cantidad máxima de espectadores permitidos.
    \item La ubicación debe ser una cadena que describa la ciudad o lugar donde se encuentra el estadio.
    \item Debe haber al menos una sección asociada a cada estadio.
    \item El nombre del estadio debe ser único.
    \item El nombre de la sección debe ser único dentro de un estadio.
    \item La capacidad de la sección debe ser un número positivo que represente la cantidad máxima de espectadores que puede albergar la sección.
    \item El número de transacción debe ser único.
    \item El cliente asociado a la transacción debe existir en la tabla Clientes.
    \item El boleto asociado a la transacción debe existir en la tabla Boletos.
    \item El precio en la transacción debe ser igual al precio del boleto multiplicado por la cantidad de boletos en la transacción.
    \item El nombre del equipo debe ser único.
    \item El país del equipo debe ser una cadena no vacía que represente el país al que pertenece el equipo.
    \item El escudo y el color de uniforme deben ser cadenas que describan la imagen del escudo y el color de uniforme del equipo, respectivamente.
    \item El id del partido debe ser único.
    \item La fecha y hora del partido deben ser válidas y estar en el futuro.
    \item Debe haber al menos dos equipos participando en cada partido.
    \item El estadio asociado al partido debe existir en la tabla Estadios.
    \item El id del equipo en el partido debe hacer referencia a un equipo existente en la tabla Equipos.
    \item El id del partido en el equipo partido debe hacer referencia a un partido existente en la tabla Partido.
    \item La combinación única de id del equipo y id del partido debe asegurar que un equipo no participe más de una vez en el mismo partido.
    \item El boleto debe tener toda la información del partido y el nombre de los dos equipos que juegan
    \item Para cada partido, por su id_Partido, solo pueden existir dos equipos en ese partido (diferentes por supuesto)
    \item Para cada partido, los colores del uniforme de los equipos no pueden ser iguales 
\end{itemize}


%%%%%----------------------------------------
%%%%%--------------------------------------
%%%%%-------------------------------------
\section{Modelo conceptual}
%%%%%-------------------------------------
%%%%%--------------------------------------
%%%%%----------------------------------------

\begin{center}
    AQUI VA EL DIAGRAMA 
\end{center}

\subsubsection*{El sistema deberá cumplir con los siguientes requisitos}
\begin{itemize}
    \item Permitir la venta de boletos por internet y en taquillas.
    \item Permitir la selección de asientos por parte de los clientes.
    \item Permitir el pago de los boletos con tarjeta de crédito, débito o efectivo.
    \item Generar reportes de ventas.
\end{itemize}

\subsubsection*{La base de datos deberá tener las siguientes tablas:}
\begin{itemize}
    \item \textbf{Boletos:}
    \begin{itemize}
        \item \textbf{NumBoleto}: El número de boleto debe ser único.
        \item \textbf{FechaPartido}: La fecha del partido debe ser una fecha válida.
        \item \textbf{Estadio}: El estadio debe ser uno de los cuatro estadios de México.
        \item \textbf{Seccion}: La sección debe ser una sección válida del estadio.
        \item \textbf{Asiento}: El asiento debe ser un asiento válido de la sección.
        \item \textbf{Precio}: El precio debe ser un número positivo.
        \item \textbf{EstadoVenta}: El estado de venta debe ser Vendido o Disponible.
    \end{itemize}

    \item \textbf{Clientes:}
    \begin{itemize}
        \item \textbf{NombreCliente}: El nombre del cliente debe ser una cadena no vacía.
        \item \textbf{Direccion}: La dirección del cliente debe ser una cadena no vacía.
        \item \textbf{Telefono}: El teléfono del cliente debe ser un número de teléfono válido.
        \item \textbf{CorreoElectronico}: El correo electrónico del cliente debe ser una dirección de correo electrónico válida.
        \item \textbf{TarjetaCredito}: La tarjeta de crédito debe ser un número de tarjeta de crédito válido.
    \end{itemize}

    \item \textbf{Estadios:}
    \begin{itemize}
        \item \textbf{CapacidadEstadio}: La capacidad del estadio debe ser un número positivo que represente la cantidad máxima de espectadores permitidos.
        \item \textbf{Ubicacion}: La ubicación debe ser una cadena que describa la ciudad o lugar donde se encuentra el estadio.
        \item \textbf{SeccionesAsociadas}: Debe haber al menos una sección asociada a cada estadio.
        \item \textbf{NombreEstadio}: El nombre del estadio debe ser único.
    \end{itemize}

    \item \textbf{Secciones:}
    \begin{itemize}
        \item \textbf{NombreSeccion}: El nombre de la sección debe ser único dentro de un estadio.
        \item \textbf{CapacidadSeccion}: La capacidad de la sección debe ser un número positivo que represente la cantidad máxima de espectadores que puede albergar la sección.
    \end{itemize}

    \item \textbf{Transacciones:}
    \begin{itemize}
        \item \textbf{NumTransaccion}: El número de transacción debe ser único.
        \item \textbf{FechaTransaccion}: La fecha de la transacción debe ser una fecha válida.
        \item \textbf{ClienteAsociado}: El cliente asociado a la transacción debe existir en la tabla Clientes.
        \item \textbf{BoletoAsociado}: El boleto asociado a la transacción debe existir en la tabla Boletos.
        \item \textbf{PrecioTransaccion}: El precio en la transacción debe ser igual al precio del boleto multiplicado por la cantidad de boletos en la transacción.
    \end{itemize}

    \item \textbf{Equipos:}
    \begin{itemize}
        \item \textbf{NombreEquipo}: El nombre del equipo debe ser único.
        \item \textbf{PaisEquipo}: El país del equipo debe ser una cadena no vacía que represente el país al que pertenece el equipo.
        \item \textbf{Escudo}: El escudo del equipo debe ser una cadena que describa la imagen del escudo del equipo.
        \item \textbf{ColorUniforme}: El color de uniforme debe ser una cadena que describa el color de uniforme del equipo.
    \end{itemize}

    \item \textbf{Partido:}
        \begin{itemize}
            \item \textbf{IdPartido}: El id del partido debe ser único.
            \item \textbf{FechaHoraPartido}: La fecha y hora del partido deben ser válidas y estar en el futuro.
            \item \textbf{EquiposParticipantes}: Debe haber al menos dos equipos participando en cada partido.
            \item \textbf{EstadioAsociado}: El estadio asociado al partido debe existir en la tabla Estadios.
            \item \textbf{EquipoLocal}: El id del equipo local en el partido debe hacer referencia a un equipo existente en la tabla Equipos.
            \item \textbf{EquipoVisitante}: El id del equipo visitante en el partido debe hacer referencia a un equipo existente en la tabla Equipos.
        \end{itemize}

    \item \textbf{EquiposPartido:}
    \begin{itemize}
        \item \textbf{IdEquipoPartido}: El id del equipo en el partido debe hacer referencia a un equipo existente en la tabla Equipos.
        \item \textbf{IdPartidoEquipo}: El id del partido en el equipo partido debe hacer referencia a un partido existente en la tabla Partido.
        \item \textbf{UnicidadEquipoPartido}: La combinación única de id del equipo y id del partido debe asegurar que un equipo no participe más de una vez en el mismo partido.
    \end{itemize}
\end{itemize}