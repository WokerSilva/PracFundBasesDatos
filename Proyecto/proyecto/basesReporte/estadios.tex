%%%%%----------------------------------------
%%%%%--------------------------------------
%%%%%-------------------------------------
\section{\texttt{Estadios}}
%%%%%-------------------------------------
%%%%%--------------------------------------
%%%%%----------------------------------------


La FIFA nos ha encargado el desarrollo de un sistema para la administración de la venta de boletos
en estadios de fútbol para los partidos del Mundial de 2026. El sistema deberá cubrir los cuatro
estadios de fútbol de México que albergarán partidos del torneo: el Estadio Azteca,
el Estadio Corregidora, el Estadio Hidalgo y el Estadio Jalisco.


\subsection{Lista de Requerimientos}

\subsubsection*{El sistema deberá cumplir con los siguientes requisitos}

\begin{itemize}
    \item Permitir la venta de boletos por internet y en taquillas.
    \item Permitir la selección de asientos por parte de los clientes.
    \item Permitir el pago de los boletos con tarjeta de crédito, débito o efectivo.
    \item Generar reportes de ventas.
\end{itemize}

\subsubsection*{La base de datos deberá tener las siguientes tablas:}

\begin{itemize}
    \item \textbf{Boletos:}
    
        Contiene los datos de los boletos, como el número de boleto, la fecha del partido,
        el estadio, la sección, el asiento, el precio y el estado de venta.

    \item \textbf{Clientes:}
    
        Contiene los datos de los clientes, como el nombre, la dirección, el teléfono, el correo electrónico y la tarjeta de crédito.

    \item \textbf{Estadios:}
    
        Contiene los datos de los estadios, como el nombre, la capacidad, la ubicación y las secciones.
    
    \item \textbf{Transacciones:} 
    
        Contiene los datos de las transacciones de venta de boletos, como el número de transacción, la fecha, el cliente, el boleto y el precio.

    \item \textbf{Equipos:} 
    
        Contiene los datos de los equipos que participan en el Mundial, como el nombre, el país, el escudo y el color de uniforme.    
\end{itemize}

\subsubsection*{Restricciones de los datos}

\begin{itemize}
    \item El número de boleto debe ser único.
    \item La fecha del partido debe ser una fecha válida.
    \item El estadio debe ser uno de los cuatro estadios de México.
    \item La sección debe ser una sección válida del estadio.
    \item El asiento debe ser un asiento válido de la sección.
    \item El precio debe ser un número positivo.
    \item El estado de venta debe ser Vendido o Disponible.
    \item El nombre del cliente debe ser una string no vacía.
    \item La dirección del cliente debe ser una string no vacía.
    \item El teléfono del cliente debe ser un número de teléfono válido.
    \item El correo electrónico del cliente debe ser una dirección de correo electrónico válida.
    \item El número de tarjeta de crédito debe ser un número de tarjeta de crédito válido.
    \item La fecha de la transacción debe ser una fecha válida.
    \item El equipo debe ser uno de los equipos participantes en el Mundial.
\end{itemize}


%%%%%----------------------------------------
%%%%%--------------------------------------
%%%%%-------------------------------------
\section{Modelo conceptual}
%%%%%-------------------------------------
%%%%%--------------------------------------
%%%%%----------------------------------------

\begin{center}
    AQUI VA EL DIAGRAMA 
\end{center}

\begin{itemize}
    \item Boletos:
    \begin{itemize}
        \item id$\_$boleto: int, llave primaria
        \item fecha$\_$partido: fecha
        \item estadio: string, llave foránea (referencia a Estadios.nombre$\_$estadio)
        \item sección: string, llave foránea (referencia a Secciones.nombre$\_$seccion)
        \item asiento: int
        \item precio: double
        \item estado$\_$venta: string
        \item fecha$\_$compra: fecha
    \end{itemize}
    \item Clientes:
    \begin{itemize}
        \item nombre$\_$cliente: string, llave primaria
        \item dirección: string
        \item teléfono: string
        \item email: string
        \item tarjeta$\_$crédito: int
    \end{itemize}
    \item Estadios:
    \begin{itemize}
        \item nombre$\_$estadio: string, llave primaria
        \item capacidad: int
        \item ubicación: string        
    \end{itemize}
    \item Secciones:
    \begin{itemize}
        \item nombre$\_$seccion: string, llave primaria
        \item id$\_$estadio: string, llave foránea (referencia a Estadios.nombre$\_$estadio)
        \item precio: double        
    \end{itemize}
    \item Transacciones:
    \begin{itemize}
        \item id$\_$transacción: int, llave primaria
        \item fecha$\_$compra: fecha
        \item nombre$\_$cliente: string,  llave foránea (referencia a Clientes.nombre$\_$cliente)
        \item id$\_$boleto: int, llave foránea (referencia a Boletos.id$\_$boleto)
        \item precio: double
        \item cantidad$\_$boletos: int
    \end{itemize}
    \item Equipos:
    \begin{itemize}
        \item nombre$\_$equipo: string, llave primaria
        \item país: string
        \item entrenador: string
        \item color$\_$uniforme: string, multivalor
        \item nombre$\_$estadio: string, llave foránea (referencia a Estadios.nombre$\_$estadio)
    \end{itemize}
    \item Partido:
    \begin{itemize}
        \item id$\_$partido: string, llave primaria
        \item fecha: fecha
        \item hora: hora
    \end{itemize}
    \item EquiposPartido:
    \begin{itemize}        
        \item id$\_$partido: string, llave foránea (referencia a Partido.id$\_$partido)
        \item nombre$\_$equipo: string, llave foránea (referencia a Equipos.nombre$\_$equipo)        
        \item rol$\_$equipo: string
        %\item UNIQUE(id$\_$partido, rol$\_$equipo)
    \end{itemize}
\end{itemize}

\subsection*{Relaciones}

Boletos $-$ Estadios: Un boleto solo puede estar relacionado con un estadio, pero un estadio puede tener muchos boletos.
Boletos $-$ Secciones: Un boleto puede estar relacionado con muchas secciones y una sección puede tener muchos boletos.
Boletos $-$ Transacciones: Un boleto puede estar relacionado con muchas transacciones y una transacción puede tener muchos boletos.
Boletos $-$ Equipos: Un boleto puede estar relacionado con uno o más equipos y un equipo puede tener muchos boletos.
Clientes$-$ Transacciones: Un cliente solo puede estar relacionado con una transacción, pero una transacción puede tener muchos clientes.
Clientes$-$ Equipos: Un cliente puede estar relacionado con uno o más equipos y un equipo puede tener muchos clientes.
Equipos $-$ Partidos: Un equipo puede estar relacionado con muchos partidos y un partido puede tener muchos equipos.


En este caso estoy en EDRPlus 
cuando agrego la sección RELATIONSHIP me da una tabla:
Entidad ONE
 Boletos: 
               one [ ]
               Many [ ]
Entidad TWO
 Estadios: 
               one [ ]
               Many [ ]

Tribuna Cabecera Local: La tribuna principal del estadio donde juegan los equipos locales.
Tribuna Cabecera Visita: La tribuna principal del estadio donde juegan los equipos visitantes.
Tribuna Lateral Visita: Una tribuna lateral del estadio donde juegan los equipos visitantes.
Tribuna Lateral Local: Una tribuna lateral del estadio donde juegan los equipos locales.
Palcos: Una sección intermedia entre la tribuna principal y la tribuna lateral.