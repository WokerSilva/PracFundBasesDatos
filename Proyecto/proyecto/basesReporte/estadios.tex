%%%%%----------------------------------------
%%%%%--------------------------------------
%%%%%-------------------------------------
\section{\texttt{Estadios}}
%%%%%-------------------------------------
%%%%%--------------------------------------
%%%%%----------------------------------------


La FIFA nos ha encargado el desarrollo de un sistema para la administración de la venta de boletos
en estadios de fútbol para los partidos del Mundial de 2026. El sistema deberá cubrir los cuatro
estadios de fútbol de México que albergarán partidos del torneo: el Estadio Azteca,
el Estadio Corregidora, el Estadio Hidalgo y el Estadio Jalisco.


\subsection{Lista de Requerimientos}

\subsubsection*{El sistema deberá cumplir con los siguientes requisitos}

\begin{itemize}
    \item Permitir la venta de boletos por internet y en taquillas.
    \item Permitir la selección de asientos por parte de los clientes.
    \item Permitir el pago de los boletos con tarjeta de crédito, débito o efectivo.
    \item Generar reportes de ventas.
\end{itemize}

\subsubsection*{La base de datos deberá tener las siguientes tablas:}

\begin{itemize}
    \item \textbf{Boletos:}
        \begin{itemize}
            \item El número de boleto debe ser único.
            \item La fecha del partido debe ser una fecha válida.
            \item El estadio debe ser uno de los cuatro estadios de México.
            \item La sección debe ser una sección válida del estadio.
            \item El asiento debe ser un asiento válido de la sección.
            \item El precio debe ser un número positivo.
            \item El estado de venta debe ser Vendido o Disponible.
        \end{itemize}

    \item \textbf{Clientes:}
        \begin{itemize}
            \item El nombre del cliente debe ser una cadena no vacía.
            \item La dirección del cliente debe ser una cadena no vacía.
            \item El teléfono del cliente debe ser un número de teléfono válido.
            \item El correo electrónico del cliente debe ser una dirección de correo electrónico válida.
            \item La tarjeta de crédito debe ser un número de tarjeta de crédito válido.
        \end{itemize}

    \item \textbf{Estadios:}
        \begin{itemize}
            \item La capacidad del estadio debe ser un número positivo que represente la cantidad máxima de espectadores permitidos.
            \item La ubicación debe ser una cadena que describa la ciudad o lugar donde se encuentra el estadio.
            \item Debe haber al menos una sección asociada a cada estadio.
            \item El nombre del estadio debe ser único.
        \end{itemize}
    \item \textbf{Secciones:}
        \begin{itemize}
            \item El nombre de la sección debe ser único dentro de un estadio.
            \item La capacidad de la sección debe ser un número positivo que represente la cantidad máxima de espectadores que puede albergar la sección.
        \end{itemize}    

    \item \textbf{Transacciones:} 
        \begin{itemize}
            \item El número de transacción debe ser único.
            \item La fecha de la transacción debe ser una fecha válida.
            \item El cliente asociado a la transacción debe existir en la tabla Clientes.
            \item El boleto asociado a la transacción debe existir en la tabla Boletos.
            \item El precio en la transacción debe ser igual al precio del boleto multiplicado por la cantidad de boletos en la transacción.
        \end{itemize}

    \item \textbf{Equipos:} 
        \begin{itemize}
            \item El nombre del equipo debe ser único.
            \item El país del equipo debe ser una cadena no vacía que represente el país al que pertenece el equipo.
            \item El escudo y el color de uniforme deben ser cadenas que describan la imagen del escudo y el color de uniforme del equipo, respectivamente.
        \end{itemize}    
    
    \item \textbf{Partido:} 
    \begin{itemize}
        \item El id del partido debe ser único.
        \item La fecha y hora del partido deben ser válidas y estar en el futuro.
        \item Debe haber al menos dos equipos participando en cada partido.
        \item El estadio asociado al partido debe existir en la tabla Estadios.
    \end{itemize}
        
    \item \textbf{EquiposPartido:} 
    \begin{itemize}
        \item El id del equipo en el partido debe hacer referencia a un equipo existente en la tabla Equipos.
        \item El id del partido en el equipo partido debe hacer referencia a un partido existente en la tabla Partido.
        \item La combinación única de id del equipo y id del partido debe asegurar que un equipo no participe más de una vez en el mismo partido.
    \end{itemize}
\end{itemize}

\subsubsection*{Restricciones de los datos}

\begin{itemize}
    \item El número de boleto debe ser único.
    \item La fecha del partido debe ser una fecha válida.
    \item El estadio debe ser uno de los cuatro estadios de México.
    \item La sección debe ser una sección válida del estadio.
    \item El asiento debe ser un asiento válido de la sección.
    \item El precio debe ser un número positivo.
    \item El estado de venta debe ser Vendido o Disponible.
    \item El nombre del cliente debe ser una cadena no vacía.
    \item La dirección del cliente debe ser una cadena no vacía.
    \item El teléfono del cliente debe ser un número de teléfono válido.
    \item El correo electrónico del cliente debe ser una dirección de correo electrónico válida.
    \item El número de tarjeta de crédito debe ser un número de tarjeta de crédito válido.
    \item La fecha de la transacción debe ser una fecha válida.
    \item El equipo debe ser uno de los equipos participantes en el Mundial.
    \item La capacidad del estadio debe ser un número positivo que represente la cantidad máxima de espectadores permitidos.
    \item La ubicación debe ser una cadena que describa la ciudad o lugar donde se encuentra el estadio.
    \item Debe haber al menos una sección asociada a cada estadio.
    \item El nombre del estadio debe ser único.
    \item El nombre de la sección debe ser único dentro de un estadio.
    \item La capacidad de la sección debe ser un número positivo que represente la cantidad máxima de espectadores que puede albergar la sección.
    \item El número de transacción debe ser único.
    \item El cliente asociado a la transacción debe existir en la tabla Clientes.
    \item El boleto asociado a la transacción debe existir en la tabla Boletos.
    \item El precio en la transacción debe ser igual al precio del boleto multiplicado por la cantidad de boletos en la transacción.
    \item El nombre del equipo debe ser único.
    \item El país del equipo debe ser una cadena no vacía que represente el país al que pertenece el equipo.
    \item El escudo y el color de uniforme deben ser cadenas que describan la imagen del escudo y el color de uniforme del equipo, respectivamente.
    \item El id del partido debe ser único.
    \item La fecha y hora del partido deben ser válidas y estar en el futuro.
    \item Debe haber al menos dos equipos participando en cada partido.
    \item El estadio asociado al partido debe existir en la tabla Estadios.
    \item El id del equipo en el partido debe hacer referencia a un equipo existente en la tabla Equipos.
    \item El id del partido en el equipo partido debe hacer referencia a un partido existente en la tabla Partido.
    \item La combinación única de id del equipo y id del partido debe asegurar que un equipo no participe más de una vez en el mismo partido.
\end{itemize}


%%%%%----------------------------------------
%%%%%--------------------------------------
%%%%%-------------------------------------
\section{Modelo conceptual}
%%%%%-------------------------------------
%%%%%--------------------------------------
%%%%%----------------------------------------

\begin{center}
    AQUI VA EL DIAGRAMA 
\end{center}

\begin{itemize}
    \item Boletos:
    \begin{itemize} 
        \item id$\_$boleto: int, llave primaria
        \item fecha$\_$partido: fecha
        \item estadio: string, llave foránea (referencia a Estadios.nombre$\_$estadio)
        \item nombre$\_$seccion: string, llave foránea (referencia a Secciones.nombre$\_$seccion)
        \item asiento: int
        \item precio: double
        \item estado$\_$venta: string
        \item fecha$\_$compra: fecha
        \item nombre$\_$equipoVisita: string llave foránea (referencia a Equipos.nombre$\_$equipo)
        \item \item nombre$\_$equipoLocal: string llave foránea (referencia a Equipos.nombre$\_$equipo)
    \end{itemize}
    \item Clientes:
    \begin{itemize}
        \item nombre$\_$cliente: string, llave primaria
        \item dirección: string
        \item teléfono: string
        \item email: string
        \item tarjeta$\_$crédito: int
    \end{itemize}
    \item Estadios:
    \begin{itemize}
        \item nombre$\_$estadio: string, llave primaria
        \item capacidad: int
        \item nombre$\_$seccion: string
        \item ubicación: string        
    \end{itemize}
    \item Secciones:
    \begin{itemize}
        \item nombre$\_$seccion: string, llave primaria
        \item id$\_$estadio: string, llave foránea (referencia a Estadios.nombre$\_$estadio)
        \item precio: double
        \item capacidad: int        
    \end{itemize}
    \item Transacciones:
    \begin{itemize}
        \item id$\_$transacción: int, llave primaria
        \item fecha$\_$compra: fecha
        \item nombre$\_$cliente: string,  llave foránea (referencia a Clientes.nombre$\_$cliente)
        \item id$\_$boleto: int, llave foránea (referencia a Boletos.id$\_$boleto)
        \item precio: double
        \item cantidad$\_$boletos: int
    \end{itemize}
    \item Equipos:
    \begin{itemize}
        \item nombre$\_$equipo: string, llave primaria
        \item país: string
        \item entrenador: string
        \item color$\_$uniforme: string, multivalor
        \item nombre$\_$estadio: string, llave foránea (referencia a Estadios.nombre$\_$estadio)
    \end{itemize}
    \item Partido:
    \begin{itemize}
        \item id$\_$partido: string, llave primaria
        \item fecha: fecha
        \item hora: hora
    \end{itemize}
    \item EquiposPartido:
    \begin{itemize}        
        \item id$\_$partido: string, llave foránea (referencia a Partido.id$\_$partido)
        \item nombre$\_$equipo: string, llave foránea (referencia a Equipos.nombre$\_$equipo)        
        \item rol$\_$equipo: string
        %\item UNIQUE(id$\_$partido, rol$\_$equipo)
    \end{itemize}
\end{itemize}

\subsection*{Relaciones}

Boletos $-$ Estadios:
Título: Venta de Boletos
Descripción: Un boleto pertenece a un estadio, y un estadio puede tener muchos boletos. Es la relación que representa la venta de boletos para un partido en un estadio específico.
Forma: Muchos a Uno (N:1)
Llaves Foráneas: estadio en Boletos hace referencia a nombre$\_$estadio en Estadios.

Boletos $-$ Secciones:
Título: Asignación a Secciones
Descripción: Un boleto está asignado a una sección, y una sección puede tener muchos boletos. Representa la asignación de asientos en una sección específica para un partido.
Forma: Muchos a Uno (N:1)
Llaves Foráneas: sección en Boletos hace referencia a nombre$\_$seccion en Secciones.

Boletos $-$ Transacciones:
Título: Registro de Compras
Descripción: Un boleto puede estar asociado con muchas transacciones, y una transacción puede tener muchos boletos. Representa la relación entre la venta de boletos y las transacciones realizadas por los clientes.
Forma: Muchos a Muchos (N:N)
Llaves Foráneas: id$\_$boleto en Transacciones hace referencia a id$\_$boleto en Boletos.

Clientes $-$ Transacciones:
Título: Historial de Compras
Descripción: Un cliente puede tener muchas transacciones, pero una transacción pertenece a un único cliente. Representa el historial de compras de un cliente.
Forma: Uno a Muchos (1:N)
Llaves Foráneas: nombre$\_$cliente en Transacciones hace referencia a nombre$\_$cliente en Clientes.

Equipos $-$ EquiposPartido:
Título: Participación en Partidos
Descripción: Un equipo puede participar en muchos partidos, y un partido puede involucrar a varios equipos. Registra la participación de equipos en distintos partidos.
Forma: Muchos a Muchos (N:N)
Llaves Foráneas: nombre$\_$equipo en EquiposPartido hace referencia a nombre$\_$equipo en Equipos, y id$\_$partido en EquiposPartido hace referencia a id$\_$partido en Partido.


EquiposPartido $-$ Partido:
Título: Relación Equipos-Partido
Descripción: EquiposPartido se relaciona con Partido para registrar la participación de equipos en partidos específicos.
Forma: Muchos a Uno (N:1)
Llaves Foráneas: id$\_$partido en EquiposPartido hace referencia a id$\_$partido en Partido.


Boletos y Partido:
Título: Asociación de Boletos a Partidos
Descripción: Un boleto está asociado a un partido, y un partido puede tener muchos boletos. Esta relación conecta la información de los boletos con los partidos en los que se utilizan.
Forma: Muchos a Uno (N:1)
Llave Foránea: id$\_$partido en Boletos hace referencia a id$\_$partido en Partido.


NOMBRE DE LOS 4 ESTADIOS
el Estadio Azteca,
el Estadio Corregidora,
el Estadio Hidalgo
el Estadio Jalisco.

SECCIONES DE UN ESTADIO
Tribuna Cabecera Local: La tribuna principal del estadio donde juegan los equipos locales.
Tribuna Cabecera Visita: La tribuna principal del estadio donde juegan los equipos visitantes.
Tribuna Lateral Visita: Una tribuna lateral del estadio donde juegan los equipos visitantes.
Tribuna Lateral Local: Una tribuna lateral del estadio donde juegan los equipos locales.
Palcos: Una sección intermedia entre la tribuna principal y la tribuna lateral.



En este caso estoy en EDRPlus 
cuando agrego la sección RELATIONSHIP me da una tabla:
Entidad ONE
 Boletos: 
               one [ ]
               Many [ ]
Entidad TWO
 Estadios: 
               one [ ]
               Many [ ]

