%%%%%----------------------------------------
%%%%%--------------------------------------
%%%%%-------------------------------------
\section{\texttt{Escuela}}
%%%%%-------------------------------------
%%%%%--------------------------------------
%%%%%----------------------------------------

%%%%%--------------------------------------
%%%%%-------------------------------------
\subsection{Lista de Requerimientos}
%%%%%-------------------------------------
%%%%%--------------------------------------


%%%%%-------------------------------------
\subsubsection*{Datos para la base de datos:}
%%%%%-------------------------------------




\subsubsection*{Restricciones de los datos}


%%%%%--------------------------------------
%%%%%-------------------------------------
\subsection{Modelo conceptual}
%%%%%-------------------------------------
%%%%%--------------------------------------
\begin{center}
    MAPA
\end{center}


%%%%%--------------------------------------
%%%%%-------------------------------------
\subsection{Modelo relacional}
%%%%%-------------------------------------
%%%%%--------------------------------------
\begin{center}
    MAPA
\end{center}


%%%%%--------------------------------------
%%%%%-------------------------------------
\subsection{Script de creación}
%%%%%-------------------------------------
%%%%%--------------------------------------
\begin{lstlisting}[caption={Tablas para la BdDatos}, label={lst:sql_estadios}]
    -- Solo los titulos de las tablas para escuela    
\end{lstlisting}



%%%%%--------------------------------------
%%%%%-------------------------------------
\subsection{Script de Insert}
%%%%%-------------------------------------
%%%%%--------------------------------------
\begin{itemize}
    \item[$\rightarrow$] Se deben generar 100 registros para cada tabla.
    \item[$\rightarrow$] Si para el buen funcionamiento de la base de datos se requieren más de 100 registros o
            menos de 100 registros en una tabla, se debe explicar claramente la razón, sólo en este caso
            sí se debe incluir un apartado en el reporte final.
\end{itemize}

Tablas:



%%%%%--------------------------------------
%%%%%-------------------------------------
\subsection{Funcionamiento restricciones}
%%%%%-------------------------------------
%%%%%--------------------------------------

Evidencia del funcionamiento de al menos 4 restricciones de integridad referencial.

\subsubsection*{Restricción 01}

\begin{itemize}
    \item[$\rightarrow$] Tablas involucradas en la restricción: 
    \item[$\rightarrow$] FK de la tabla que referencia y PK de la tabla referenciada: 
    \item[$\rightarrow$] Justificación del trigger de integridad referencial elegido: 
    \item[$\rightarrow$] Instrucción UPDATE o DELETE que permita evidenciar que la restricción está
    funcionando.
    \begin{verbatim}
    INSTRUCCION
    \end{verbatim}
    \item[$\rightarrow$] Captura de pantalla con el resultado de la instrucción que muestre que la restricción está
    funcionando.
    \begin{center}
        CAPTURA
    \end{center}
\end{itemize}

\subsubsection*{Restricción 01}

\begin{itemize}
    \item[$\rightarrow$] Tablas involucradas en la restricción: 
    \item[$\rightarrow$] FK de la tabla que referencia y PK de la tabla referenciada: 
    \item[$\rightarrow$] Justificación del trigger de integridad referencial elegido: 
    \item[$\rightarrow$] Instrucción UPDATE o DELETE que permita evidenciar que la restricción está
    funcionando.
    \begin{verbatim}
    INSTRUCCION
    \end{verbatim}
    \item[$\rightarrow$] Captura de pantalla con el resultado de la instrucción que muestre que la restricción está
    funcionando.
    \begin{center}
        CAPTURA
    \end{center}
\end{itemize}

\subsubsection*{Restricción 02}

\begin{itemize}
    \item[$\rightarrow$] Tablas involucradas en la restricción: 
    \item[$\rightarrow$] FK de la tabla que referencia y PK de la tabla referenciada: 
    \item[$\rightarrow$] Justificación del trigger de integridad referencial elegido: 
    \item[$\rightarrow$] Instrucción UPDATE o DELETE que permita evidenciar que la restricción está
    funcionando.
    \begin{verbatim}
    INSTRUCCION
    \end{verbatim}
    \item[$\rightarrow$] Captura de pantalla con el resultado de la instrucción que muestre que la restricción está
    funcionando.
    \begin{center}
        CAPTURA
    \end{center}
\end{itemize}

\subsubsection*{Restricción 03}

\begin{itemize}
    \item[$\rightarrow$] Tablas involucradas en la restricción: 
    \item[$\rightarrow$] FK de la tabla que referencia y PK de la tabla referenciada: 
    \item[$\rightarrow$] Justificación del trigger de integridad referencial elegido: 
    \item[$\rightarrow$] Instrucción UPDATE o DELETE que permita evidenciar que la restricción está
    funcionando.
    \begin{verbatim}
    INSTRUCCION
    \end{verbatim}
    \item[$\rightarrow$] Captura de pantalla con el resultado de la instrucción que muestre que la restricción está
    funcionando.
    \begin{center}
        CAPTURA
    \end{center}
\end{itemize}

\subsubsection*{Restricción 04}

\begin{itemize}
    \item[$\rightarrow$] Tablas involucradas en la restricción: 
    \item[$\rightarrow$] FK de la tabla que referencia y PK de la tabla referenciada: 
    \item[$\rightarrow$] Justificación del trigger de integridad referencial elegido: 
    \item[$\rightarrow$] Instrucción UPDATE o DELETE que permita evidenciar que la restricción está
    funcionando.
    \begin{verbatim}
    INSTRUCCION
    \end{verbatim}
    \item[$\rightarrow$] Captura de pantalla con el resultado de la instrucción que muestre que la restricción está
    funcionando.
    \begin{center}
        CAPTURA
    \end{center}
\end{itemize}


%%%%%--------------------------------------
%%%%%-------------------------------------
\subsection{Funcionamiento Restricciones check}
%%%%%-------------------------------------
%%%%%--------------------------------------

Evidencia del funcionamiento de al menos 3 restricciones check para “atributos” de varias
tablas.

\begin{itemize}
    \item Tabla elegida
    \item Atributo elegido
    \item Breve descripción de la restricción
    \item Instrucción para la creación de la restricción.
    \item Instrucción que permita evidenciar que la restricción esta funcionando.
    \item Captura de pantalla con el resultado de la instrucción que muestre que la restricción está
    funcionando.
\end{itemize}




%%%%%--------------------------------------
%%%%%-------------------------------------
\subsection{Creación de dominios personalizados}
%%%%%-------------------------------------
%%%%%--------------------------------------

Evidencia de la creación de al menos tres dominios personalizados. Se deben utilizar restricciones check en la creación de los tres dominios.
\begin{itemize}
    \item Tabla elegida
    \item Atributo elegido
    \item Breve descripción del dominio y de la restricción check propuesta.
    \item Instrucción para la creación del dominio personalizado.
    \item Captura de pantalla de la estructura de la tabla donde se muestre el dominio personalizado
    en uso.
\end{itemize}



%%%%%--------------------------------------
%%%%%-------------------------------------
\subsection{Restricciones para tuplas}
%%%%%-------------------------------------
%%%%%--------------------------------------

Evidencia del funcionamiento de al menos 2 restricciones para “tuplas” en diferentes tablas (Unidad 8 Integridad, tema Specifying Constraints on Tuples Using CHECK)
\begin{itemize}
    \item Tabla elegida
    \item Breve descripción de la restricción.
    \item Instrucción para la creación de la restricción.
    \item Instrucción Insert o Update que permita evidenciar que la restricción esta funcionando.
    \item Captura de pantalla con el resultado de la instrucción que muestre que la restricción está
    funcionando.
\end{itemize}



%%%%%--------------------------------------
%%%%%-------------------------------------
\subsection{Consultas}
%%%%%-------------------------------------
%%%%%--------------------------------------

Plantea 3 consultas que consideres relevantes para la base de datos propuesta. Para cada consulta planteada, incluir en el reporte los siguientes incisos:
\begin{itemize}
    \item Redacción clara de la consulta.
    \item Código en lenguaje SQL de la consulta.
    \item Ejecutar la consulta en Postgres e incluir una captura de pantalla con el resultado de la
    consulta.
\end{itemize}

%%%%%--------------------------------------
%%%%%-------------------------------------
\subsection{Vistas}
%%%%%-------------------------------------
%%%%%--------------------------------------

Plantea 3 vistas que consideres relevantes para la base de datos propuesta. Para cada vista planteada, incluir en el reporte los siguientes incisos:
\begin{itemize}
    \item Redacción clara de la vista planteada.
    \item Código en lenguaje SQL que permita crear la vista solicitada.
    \item Ejecutar el código para la creación de la vista en Postgres e incluir una captura de pantalla
    con la vista creada satisfactoriamente.
    \item Incluir un ejemplo que los evaluadores puedan ejecutar para verificar el funcionamiento
    de las vistas.
\end{itemize}