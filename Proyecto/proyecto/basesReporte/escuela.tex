%%%%%----------------------------------------
%%%%%--------------------------------------
%%%%%-------------------------------------
\section{\texttt{Escuela}}
%%%%%-------------------------------------
%%%%%--------------------------------------
%%%%%----------------------------------------
  
\subsection{Requerimientos}

Saludos desde Hogwarts Escuela de Magia y Hechicería. Después de revisar exhaustivamente sus habilidades y
experiencia en el mundo mágico de la programación, nos complace informarles que han sido seleccionados para 
crear y gestionar la base de datos mágica de nuestra venerable institución.

\subsubsection*{Principales restricciones}
\begin{itemize}
    \item La clave primaria en cada tabla debe ser única y no nula.
    \item Las claves foráneas deben hacer referencia a claves primarias existentes en las tablas referenciadas.
    \item La columna de Fecha de Nacimiento en las tablas Alumnos y Maestros debe contener fechas válidas.
    \item La columna de Género en las tablas Alumnos y Maestros debe contener valores específicos: 'Masculino' o 'Femenino'.
    \item La columna de Casa en la tabla Alumnos debe contener valores específicos: 'Gryffindor', 'Hufflepuff', 'Ravenclaw' o 'Slytherin'.
    \item La columna de Año Académico en la tabla Cursos debe contener valores específicos: '1ro', '2do', '3ro', '4to', '5to', '6to'.    
    \item La columna de Nombre y Apellido en las tablas Alumnos y Maestros debe contener texto no nulo y no vacío.
    \item La tabla Asignaturas debe tener una relación adecuada entre las claves foráneas y la información del profesor y el curso.
    \item La columna de Localidad en la tabla Ubicacion debe contener valores no nulos y no vacíos.  
    \item Cada año escolar tiene un curso asociado 
\end{itemize}

%%%%%--------------------------------------
%%%%%-------------------------------------
\subsection{Modelo conceptual}
%%%%%-------------------------------------
%%%%%--------------------------------------
\begin{center}
    \includegraphics[scale = .7]{IMA/escuela/BD-ESCUELA-MO-CONCEPTUAL.png}
\end{center}


%%%%%--------------------------------------
%%%%%-------------------------------------
\subsection{Modelo relacional}
%%%%%-------------------------------------
%%%%%--------------------------------------
\begin{center}
  \includegraphics[scale = .7]{IMA/escuela/BD-ESCUELA-MO-RELACIONAL.png}
\end{center}


%%%%%--------------------------------------
%%%%%-------------------------------------
\subsection{Script de creación}
%%%%%-------------------------------------
%%%%%--------------------------------------
\begin{lstlisting}[caption={Tablas para la BdDatos}, label={lst:sql_estadios}]
    -- Solo los titulos de las tablas para escuela    
    CREATE TABLE UBICACION
    (
      
    );
    
    CREATE TABLE MAESTROS
    (
      
    );
    
    CREATE TABLE CURSOS
    (
      
    );
    
    CREATE TABLE ASIGNATURAS
    (
      
    );
    
    CREATE TABLE ALUMNOS
    (

    );    
\end{lstlisting}

%%%%%--------------------------------------
%%%%%-------------------------------------
\subsection{Script de Insert}
%%%%%-------------------------------------
%%%%%--------------------------------------
\begin{itemize}
    \item[$\rightarrow$] Se deben generar 100 registros para cada tabla.
    \item[$\rightarrow$] Si para el buen funcionamiento de la base de datos se requieren más de 100 registros o
            menos de 100 registros en una tabla, se debe explicar claramente la razón, sólo en este caso
            sí se debe incluir un apartado en el reporte final.
\end{itemize}

Tablas:
\begin{itemize}
  \item UBICACION: Si es posible llegar a 100 registros de ubicación 
  \item MAESTROS: No es posible llegar a 100 porque la tabla solo contempla 24 maestros registrados
  \item CURSOS: No es posible llegar a 100 porque solo se contemplan 6 cursos para la escuela
  \item ASIGNATURAS: No es posible llegar a 100 porque hay 7 asignaturas por cada curso, son 24 asignaturas
  \item Alumnos: Si es posible llegar a 100 registros de alumnos
\end{itemize}

%%%%%--------------------------------------
%%%%%-------------------------------------
\subsection{Funcionamiento Restricciones}
%%%%%-------------------------------------
%%%%%--------------------------------------

Evidencia del funcionamiento de al menos 4 restricciones de integridad referencial.

\subsubsection*{Restricción 01}

\begin{itemize}
    \item[$\rightarrow$] Tablas involucradas en la restricción: ALUMNOS y UBICACION
    \item[$\rightarrow$] FK de la tabla que referencia y PK de la tabla referenciada: ubicacionID en ALUMNOS es una clave foránea que referencia a ubicacionID en UBICACION
    \item[$\rightarrow$] Justificación del trigger de integridad referencial elegido: Esta restricción asegura que cada alumno esté asociado con una ubicación válida en la tabla UBICACION. Si se intenta eliminar una ubicación que está siendo referenciada por un alumno, o se intenta actualizar el ubicacionID de una ubicación a un valor que no existe en la tabla ALUMNOS, la operación será rechazada.
    \item[$\rightarrow$] Instrucción UPDATE o DELETE que permita evidenciar que la restricción está funcionando.
    \begin{verbatim}
      DELETE FROM UBICACION WHERE ubicacionID = 'OXFU';
    \end{verbatim}
    \item[$\rightarrow$] Captura de pantalla con el resultado de la instrucción que muestre que la restricción está
    funcionando.
    \begin{center}
        \includegraphics[scale = .4]{IMA/escuela/H-6-01.png}
    \end{center}
\end{itemize}


\subsubsection*{Restricción 02}

\begin{itemize}
  \item[$\rightarrow$] Tablas involucradas en la restricción: CURSOS y MAESTROS
  \item[$\rightarrow$] FK de la tabla que referencia y PK de la tabla referenciada: maestroID en CURSOS es una clave foránea que referencia a maestroID en MAESTROS
  \item[$\rightarrow$] Justificación del trigger de integridad referencial elegido: Esta restricción asegura que cada curso esté asociado con un maestro válido en la tabla MAESTROS. Si se intenta eliminar un maestro que está siendo referenciado por un curso, o se intenta actualizar el maestroID de un maestro a un valor que no existe en la tabla CURSOS, la operación será rechazada.
  \item[$\rightarrow$] Instrucción UPDATE o DELETE que permita evidenciar que la restricción está funcionando.
  \begin{verbatim}
    DELETE FROM MAESTROS WHERE maestroID = 'GGO01';
  \end{verbatim}
  \item[$\rightarrow$] Captura de pantalla con el resultado de la instrucción que muestre que la restricción está funcionando.
  \begin{center}
    \includegraphics[scale = .4]{IMA/escuela/H-6-02.png}
  \end{center}
\end{itemize}

\subsubsection*{Restricción 03}

\begin{itemize}
  \item[$\rightarrow$] Tablas involucradas en la restricción: ASIGNATURAS y CURSOS
  \item[$\rightarrow$] FK de la tabla que referencia y PK de la tabla referenciada: cursoID en ASIGNATURAS es una clave foránea que referencia a cursoID en CURSOS
  \item[$\rightarrow$] Justificación del trigger de integridad referencial elegido: Esta restricción asegura que cada asignatura esté asociada con un curso válido en la tabla CURSOS. Si se intenta eliminar un curso que está siendo referenciado por una asignatura, o se intenta actualizar el cursoID de un curso a un valor que no existe en la tabla ASIGNATURAS, la operación será rechazada.
  \item[$\rightarrow$] Instrucción UPDATE o DELETE que permita evidenciar que la restricción está funcionando.
  \begin{verbatim}
    DELETE FROM CURSOS WHERE cursoID = 'A1IM';
  \end{verbatim}
  \item[$\rightarrow$] Captura de pantalla con el resultado de la instrucción que muestre que la restricción está funcionando.
  \begin{center}
    \includegraphics[scale = .4]{IMA/escuela/H-6-03.png}
  \end{center}
\end{itemize}


\subsubsection*{Restricción 03}

\begin{itemize}
  \item[$\rightarrow$] Tablas involucradas en la restricción: ALUMNOS y CURSOS
  \item[$\rightarrow$] FK de la tabla que referencia y PK de la tabla referenciada: cursoID en ALUMNOS es una clave foránea que referencia a cursoID en CURSOS
  \item[$\rightarrow$] Justificación del trigger de integridad referencial elegido: Esta restricción asegura que cada alumno esté asociado con un curso válido en la tabla CURSOS. Si se intenta eliminar un curso que está siendo referenciado por un alumno, o se intenta actualizar el cursoID de un curso a un valor que no existe en la tabla ALUMNOS, la operación será rechazada.
  \item[$\rightarrow$] Instrucción UPDATE o DELETE que permita evidenciar que la restricción está funcionando.
  \begin{verbatim}
    UPDATE ALUMNOS
      SET cursoID = 'A7SA'
      WHERE alumnoID = 'S29';
  \end{verbatim}
  \item[$\rightarrow$] Captura de pantalla con el resultado de la instrucción que muestre que la restricción está funcionando.
  \begin{center}
    \includegraphics[scale = .4]{IMA/escuela/H-6-04.png}
  \end{center}
\end{itemize}

%%%%%--------------------------------------
%%%%%-------------------------------------
\subsection{Funcionamiento Restricciones check}
%%%%%-------------------------------------
%%%%%--------------------------------------

Evidencia del funcionamiento de al menos 3 restricciones check para “atributos” de varias
tablas.

\begin{itemize} 
  \item Tabla elegida: ALUMNOS 
  \item Atributo elegido: casa 
  \item Descripción: Esta restricción asegura que el valor de la columna Casa sea uno de los cuatro nombres de las casas de Hogwarts: 'Hufflepuff', 'Ravenclaw', 'Gryffindor', 'Slytherin'. 
  \item Instrucción para la creación de la restricción: 
    \begin{verbatim} 
    ALTER TABLE ALUMNOS 
    ADD CONSTRAINT casaCasa CHECK (Casa IN ('Hufflepuff',
                                            'Ravenclaw', 
                                            'Gryffindor', 
                                            'Slytherin')); 
    \end{verbatim} 
  \item Instrucción que permita evidenciar que la restricción esta funcionando: 
    \begin{verbatim} 
    INSERT INTO ALUMNOS (alumnoID, Nombre, Apellido, Casa, fechaNacimiento,
                         Genero, ubicacionID, cursoID) 
      VALUES ('nuevoID', 'Nombre', 'Apellido', 'CasaIncorrecta', 
              '2000-01-01', 'Genero', 'ubicacionID', 'cursoID'); 
    \end{verbatim} 
  \item Captura de pantalla 
    \begin{center} 
      \includegraphics[scale = .4]{IMA/escuela/H-7-01.png}
    \end{center} 
\end{itemize}

\begin{itemize} 
  \item Tabla elegida: MAESTROS 
  \item Atributo elegido: fechaNacimiento 
  \item Descripción: Esta restricción asegura que el valor de la columna fechaNacimiento sea una fecha válida y no sea en el futuro. 
  \item Instrucción para la creación de la restricción: 
  \begin{verbatim} 
    ALTER TABLE MAESTROS 
    ADD CONSTRAINT fechaMaestro CHECK (fechaNacimiento <= CURRENT_DATE); 
  \end{verbatim} 
  \item Instrucción que permita evidenciar que la restricción esta funcionando: 
  \begin{verbatim} 
    INSERT INTO MAESTROS 
        (maestroID, Nombre, Apellido, Especialidad, fechaNacimiento, 
         Genero, ubicacionID) 
      VALUES (‘nuevoID’, ‘Nombre’, ‘Apellido’, ‘Especialidad’,
              ‘3000-01-01’, ‘Genero’, ‘ubicacionID’); 
  \end{verbatim}   
\end{itemize}

\begin{itemize} 
  \item Tabla elegida: ALUMNOS 
  \item Atributo elegido: fechaNacimiento 
  \item Breve descripción de la restricción: Esta restricción asegura que el valor de la columna fechaNacimiento sea una fecha válida y no sea en el futuro.
  \item Instrucción para la creación de la restricción: 
  \begin{verbatim} 
  ALTER TABLE ALUMNOS 
  ADD CONSTRAINT fechaAlumno CHECK (fechaNacimiento <= CURRENT_DATE); 
  \end{verbatim} 
  \item Instrucción que permita evidenciar que la restricción esta funcionando: 
  \begin{verbatim} 
  INSERT INTO ALUMNOS (alumnoID, Nombre, Apellido, Casa, fechaNacimiento, 
                       Genero, ubicacionID, cursoID) 
    VALUES (‘nuevoID’, ‘Nombre’, ‘Apellido’, ‘Hufflepuff’, ‘3000-01-01’, 
            ‘Genero’, ‘ubicacionID’, ‘cursoID’); 
  \end{verbatim} 
\end{itemize}


%%%%%--------------------------------------
%%%%%-------------------------------------
\subsection{Creación de dominios personalizados}
%%%%%-------------------------------------
%%%%%--------------------------------------

Evidencia de la creación de al menos tres dominios personalizados. Se deben utilizar restricciones check en la creación de los tres dominios.
\begin{itemize}
    \item Tabla elegida: ALUMNOS
    \item Atributo elegido: Casa
    \item Descripción: El dominio Casa representa las cuatro casas de la escuela Hogwarts de Magia y Hechicería. Las cuatro casas son: Hufflepuff Ravenclaw Gryffindor Slytherin
    \item Instrucción para la creación del dominio personalizado.
    \begin{lstlisting}[caption={Tablas para la BdDatos}, label={lst:sql_estadios}]
      CREATE DOMAIN Casa
        AS CHAR(20)
        CHECK (
          Casa IN ('Hufflepuff', 'Ravenclaw', 
                   'Gryffindor', 'Slytherin')
        );      
    \end{lstlisting}
\end{itemize}


\begin{itemize}
    \item Tabla elegida: MAESTROS
    \item Atributo elegido: genero
    \item Descripción: El dominio Genero representa el género de un maestro. Los valores permitidos son: Masculino, Femenino
    \item Instrucción para la creación del dominio personalizado.
    
    \begin{lstlisting}[caption={Tablas para la BdDatos}, label={lst:sql_estadios}]
      CREATE DOMAIN Genero
      AS CHAR(20)
      CHECK (
        Genero IN ('Masculino', 'Femenino')
      );      
    \end{lstlisting}    
\end{itemize}


\begin{itemize}
    \item Tabla elegida: MAESTROS
    \item Atributo elegido: especialidad
    \item Descripción: El dominio Especialidad representa la especialidad de un maestro. La restricción check propuesta indica que el valor de este atributo debe ser una cadena de caracteres de al menos 3 caracteres y no más de 50 caracteres.
    \item Instrucción para la creación del dominio personalizado.
    
    \begin{lstlisting}[caption={Tablas para la BdDatos}, label={lst:sql_estadios}]
      USE [EscuelaDeMagia]
      GO

      CREATE DOMAIN [Especialidad] AS VARCHAR(50)
      CHECK (LENGTH(VALUE) >= 3 AND LENGTH(VALUE) <= 50)
      GO
    \end{lstlisting}    
\end{itemize}



%%%%%--------------------------------------
%%%%%-------------------------------------
\subsection{Restricciones para tuplas}
%%%%%-------------------------------------
%%%%%--------------------------------------

Evidencia del funcionamiento de al menos 2 restricciones para “tuplas” en diferentes tablas (Unidad 8 Integridad, tema Specifying Constraints on Tuples Using CHECK)

\begin{itemize}
    \item Tabla elegida: ALUMNOS
    \item Breve descripción de la restricción: Esta restricción asegura que los alumnos no puedan ser mayores de 100 años. Para ello, se verifica que la fecha de nacimiento sea posterior al año actual menos 100.
    \item Instrucción para la creación de la restricción:
    \begin{verbatim}
    ALTER TABLE ALUMNOS
    ADD CONSTRAINT edadAlumno
    CHECK (fechaNacimiento > DATE_SUB(CURRENT_DATE, INTERVAL 100 YEAR));
    \end{verbatim}
    \item Instrucción Insert o Update que permita evidenciar que la restricción esta funcionando:
    \begin{verbatim}
    INSERT INTO ALUMNOS (alumnoID, Nombre, Apellido, Casa, fechaNacimiento, Genero, ubicacionID, cursoID)
    VALUES ('nuevoID', 'Nombre', 'Apellido', 'Hufflepuff', '1900-01-01', 'Genero', 'ubicacionID', 'cursoID');
    \end{verbatim}  
\end{itemize}


\begin{itemize}
  \item Tabla elegida: `MAESTROS`
  \item Breve descripción de la restricción: Esta restricción asegura que los maestros no puedan ser menores de 25 años. Para ello, se verifica que la fecha de nacimiento sea anterior al año actual menos 25.
  \item Instrucción para la creación de la restricción:
  \begin{verbatim}
  ALTER TABLE MAESTROS
  ADD CONSTRAINT edadMaestro
  CHECK (fechaNacimiento < DATE_SUB(CURRENT_DATE, INTERVAL 25 YEAR));
  \end{verbatim}
  \item Instrucción Insert o Update que permita evidenciar que la restricción esta funcionando:
  \begin{verbatim}
  INSERT INTO MAESTROS (maestroID, Nombre, Apellido, Especialidad, fechaNacimiento, Genero, ubicacionID)
  VALUES ('nuevoID', 'Nombre', 'Apellido', 'Especialidad', '2000-01-01', 'Genero', 'ubicacionID');
  \end{verbatim}
\end{itemize}

%%%%%--------------------------------------
%%%%%-------------------------------------
\subsection{Consultas}
%%%%%-------------------------------------
%%%%%--------------------------------------

Plantea 3 consultas que consideres relevantes para la base de datos propuesta. Para cada consulta planteada, incluir en el reporte los siguientes incisos:

\subsubsection*{Consulta 01}
\begin{itemize}
    \item Redacción clara de la consulta: Obtener una lista de todos los alumnos que están en la casa Gryffindor.
    \item Código en lenguaje SQL de la consulta.
    
    \begin{lstlisting}[caption={Tablas para la BdDatos}, label={lst:sql_estadios}]
      SELECT Nombre, Apellido
      FROM ALUMNOS
      WHERE Casa = 'Slytherin';
    \end{lstlisting}    

    \begin{center}
      \includegraphics[scale = .5]{IMA/escuela/consu01.png}
    \end{center}
    
\end{itemize}


\subsubsection*{Consulta 02}
\begin{itemize}
    \item Redacción clara de la consulta: Obtener una lista de todos los cursos que son impartidos por un maestro específico.
    \item Código en lenguaje SQL de la consulta.
    
    \begin{lstlisting}[caption={Tablas para la BdDatos}, label={lst:sql_estadios}]
      SELECT nombreCurso
      FROM CURSOS
      WHERE maestroID = 'GGO01';
    \end{lstlisting}    

    \begin{center}
      \includegraphics[scale = .8]{IMA/escuela/consu02.png}
    \end{center}
    
\end{itemize}


\subsubsection*{Consulta 03}
\begin{itemize}
    \item Redacción clara de la consulta:  Obtener una lista de todos los maestros que viven en una ubicación específica.
    \item Código en lenguaje SQL de la consulta.
    
    \begin{lstlisting}[caption={Tablas para la BdDatos}, label={lst:sql_estadios}]
      SELECT Nombre, Apellido
      FROM MAESTROS
      WHERE ubicacionID = 'NEWC';
    \end{lstlisting}    

    \begin{center}
      \includegraphics[scale = .8]{IMA/escuela/consu03.png}
    \end{center}

\end{itemize}

%%%%%--------------------------------------
%%%%%-------------------------------------
\subsection{Vistas}
%%%%%-------------------------------------
%%%%%--------------------------------------

Plantea 3 vistas que consideres relevantes para la base de datos propuesta. Para cada vista planteada, incluir en el reporte los siguientes incisos:

\subsection*{Vista 01}
\begin{itemize}
    \item Redacción clara de la vista planteada: Una vista que muestre todos los alumnos junto con la información de su ubicación.
    \item Código en lenguaje SQL que permita crear la vista solicitada.
    
    \begin{lstlisting}[caption={Tablas para la BdDatos}, label={lst:sql_estadios}]
      CREATE VIEW AlumnosUbicacion AS
      SELECT ALUMNOS.Nombre, ALUMNOS.Apellido, UBICACION.Localidad, UBICACION.Pais
      FROM ALUMNOS
      JOIN UBICACION ON ALUMNOS.ubicacionID = UBICACION.ubicacionID;
    \end{lstlisting}    
    
    \item USO:
    
    \begin{lstlisting}[caption={Tablas para la BdDatos}, label={lst:sql_estadios}]
      SELECT * FROM AlumnosUbicacion;
    \end{lstlisting}    

    \begin{center}
      \includegraphics[scale = .4]{IMA/escuela/vista01.png}
    \end{center}
\end{itemize}


\subsection*{Vista 02}

\begin{itemize}
  \item Redacción clara de la vista planteada: Una vista que muestre todos los cursos junto con el nombre del maestro que los imparte.
  \item Código en lenguaje SQL que permita crear la vista solicitada.
  
  \begin{lstlisting}[caption={Tablas para la BdDatos}, label={lst:sql_estadios}]
    CREATE VIEW CursosMaestros AS
    SELECT CURSOS.nombreCurso, MAESTROS.Nombre, MAESTROS.Apellido
    FROM CURSOS
    JOIN MAESTROS ON CURSOS.maestroID = MAESTROS.maestroID;
  \end{lstlisting}    
  
  \item USO:
  
  \begin{lstlisting}[caption={Tablas para la BdDatos}, label={lst:sql_estadios}]
    SELECT * FROM CursosMaestros;
  \end{lstlisting}    

  \begin{center}
    \includegraphics[scale = .4]{IMA/escuela/vista02.png}
  \end{center}
\end{itemize}


\subsection*{Vista 03}

\begin{itemize}
  \item Redacción clara de la vista planteada: Una vista que muestre todas las asignaturas junto con el nombre del curso al que pertenecen.
  \item Código en lenguaje SQL que permita crear la vista solicitada.
  
  \begin{lstlisting}[caption={Tablas para la BdDatos}, label={lst:sql_estadios}]
    CREATE VIEW AsignaturasCursos AS
    SELECT ASIGNATURAS.nombreAsignatura, CURSOS.nombreCurso
    FROM ASIGNATURAS
    JOIN CURSOS ON ASIGNATURAS.cursoID = CURSOS.cursoID;
  \end{lstlisting}    
  
  \item USO:
  
  \begin{lstlisting}[caption={Tablas para la BdDatos}, label={lst:sql_estadios}]
    SELECT * FROM AsignaturasCursos;
  \end{lstlisting}    

  \begin{center}
    \includegraphics[scale = .4]{IMA/escuela/vista03.png}
  \end{center}
\end{itemize}