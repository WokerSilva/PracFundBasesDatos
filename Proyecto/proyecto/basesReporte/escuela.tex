%%%%%----------------------------------------
%%%%%--------------------------------------
%%%%%-------------------------------------
\section{\texttt{Escuela}}
%%%%%-------------------------------------
%%%%%--------------------------------------
%%%%%----------------------------------------

%%%%%--------------------------------------
%%%%%-------------------------------------
\subsection{Lista de Requerimientos}
%%%%%-------------------------------------
%%%%%--------------------------------------

\subsubsection*{Tablas para la base de datos}

\begin{itemize}
    \item Tabla Alumnos:
      \begin{itemize}
        \item \textit{AlumnoID} (Clave primaria)
        \item \textit{Nombre}
        \item \textit{Apellido}
        \item \textit{Casa} 
        \item \textit{Fecha de Nacimiento}
        \item \textit{Género}
        \item \textit{CursoID} (Clave foránea referenciando a la tabla Cursos)
        \item \textit{UbicacionID} (Clave foránea referenciando a la tabla Ubicacion)
      \end{itemize}
    \item Tabla Maestros:
      \begin{itemize}
        \item \textit{maestroID} (Clave primaria)
        \item \textit{Nombre}
        \item \textit{Apellido}
        \item \textit{Especialidad}
        \item \textit{Fecha de Nacimiento}
        \item \textit{Género}
        \item \textit{UbicacionID} (Clave foránea referenciando a la tabla Ubicacion)
      \end{itemize}
    \item Tabla Ubicacion:
      \begin{itemize}
        \item \textit{UbicacionID} (Clave primaria)
        \item \textit{Localidad}
        \item \textit{Pais}
      \end{itemize}
    \item Tabla Cursos:
      \begin{itemize}
        \item \textit{CursoID} (Clave primaria)
        \item \textit{Nombre del Curso}
        \item \textit{gradoAcademico} 
        \item \textit{ProfesorJefeID} (Clave foránea referenciando a la tabla Maestros)
      \end{itemize}
    \item Tabla Asignaturas:
      \begin{itemize}
        \item \textit{AsignaturaID} (Clave primaria)
        \item \textit{Nombre de la Asignatura}
        \item \textit{maestroID} (Clave foránea referenciando a la tabla Maestros)
        \item \textit{CursoID} (Clave foránea referenciando a la tabla Cursos)
      \end{itemize}
  \end{itemize}

\subsubsection*{Principales restricciones}
\begin{itemize}
    \item La clave primaria en cada tabla debe ser única y no nula.
    \item Las claves foráneas deben hacer referencia a claves primarias existentes en las tablas referenciadas.
    \item La columna de Fecha de Nacimiento en las tablas Alumnos y Maestros debe contener fechas válidas.
    \item La columna de Género en las tablas Alumnos y Maestros debe contener valores específicos: 'Masculino' o 'Femenino'.
    \item La columna de Casa en la tabla Alumnos debe contener valores específicos: 'Gryffindor', 'Hufflepuff', 'Ravenclaw' o 'Slytherin'.
    \item La columna de Año Académico en la tabla Cursos debe contener valores específicos: '1ro', '2do', '3ro', '4to', '5to', '6to'.    
    \item La columna de Nombre y Apellido en las tablas Alumnos y Maestros debe contener texto no nulo y no vacío.
    \item La tabla Asignaturas debe tener una relación adecuada entre las claves foráneas y la información del profesor y el curso.
    \item La columna de Localidad en la tabla Ubicacion debe contener valores no nulos y no vacíos.  
    \item Cada año escolar tiene un curso asociado 
\end{itemize}

\subsubsection*{Información de la escuela}

\begin{itemize}
  \item Curso Primer Año:  Iniciación Mágica
    \begin{itemize}
      \item \textit{Encantamientos Básicos}
      \item \textit{Herbología Elemental}
      \item \textit{Pociones para Principiantes}
      \item \textit{Historia de la Magia Antigua}
      \item \textit{Defensa Básica contra las Artes Oscuras}
      \item \textit{Vuelo Inicial en Escoba}
      \item \textit{Estudios Muggles Introductorios}
    \end{itemize}
  \item Curso Segundo Año: Desarrollo Mágico
    \begin{itemize}
      \item \textit{Transformaciones Intermedias}
      \item \textit{Cuidado de Criaturas Mágicas}
      \item \textit{Pociones Avanzadas}
      \item \textit{Historia de la Magia Contemporánea}
      \item \textit{Defensa Intermedia contra las Artes Oscuras}
      \item \textit{Aritmancia Elemental}
      \item \textit{Runas Antiguas}
    \end{itemize}
  \item Curso Tercer Año:  Magia Avanzada
    \begin{itemize}
      \item \textit{Encantamientos Avanzados}
      \item \textit{Herbología Avanzada}
      \item \textit{Pociones Maestras}
      \item \textit{Historia de la Magia Medieval}
      \item \textit{Defensa Avanzada contra las Artes Oscuras}
      \item \textit{Adivinación Elemental}
      \item \textit{Astronomía Mágica}
    \end{itemize}
  \item Curso Cuarto Año: Especialización Mágica
    \begin{itemize}
      \item \textit{Transformaciones Avanzadas}
      \item \textit{Cuidado de Criaturas Mágicas Exóticas}
      \item \textit{Pociones Maestras Avanzadas}
      \item \textit{Historia de la Magia Moderna}
      \item \textit{Defensa contra las Artes Oscuras Especializada}
      \item \textit{Estudios de Artefactos Mágicos}
      \item \textit{Ocultismo}
    \end{itemize}
  \item 
    \begin{itemize}
      \item \textit{Encantamientos Supremos}
      \item \textit{Herbología Aplicada}
      \item \textit{Pociones Maestras Maestría}
      \item \textit{Historia de la Magia Contemporánea Avanzada}
      \item \textit{Defensa Avanzada contra las Artes Oscuras Aplicada}
      \item \textit{Aritmancia Avanzada}
      \item \textit{Estudio de las Maldiciones}
    \end{itemize}
  \item Curso Sexto Año: Sabiduría Arcana
    \begin{itemize}
      \item \textit{Transformaciones Maestras}
      \item \textit{Cuidado de Criaturas Mágicas Legendarias}
      \item \textit{Pociones Maestras Supremas}
      \item \textit{Historia de la Magia Postmoderna}
      \item \textit{Defensa Maestra contra las Artes Oscuras}
      \item \textit{Adivinación Avanzada}
      \item \textit{Astronomía Avanzada}
    \end{itemize}
\end{itemize}

\begin{table}[H]
  \centering
  \begin{tabular}{c|c|c}
  \textbf{Curso} & \textbf{Asignatura} & \textbf{Profesor} \\ \hline
  \multirow{2}{*}{Primer Año} & Iniciación Mágica & \\
  & Encantamientos Básicos & Profesor Filius Flitwick \\
  & Herbología Elemental & Profesora Pomona Sprout \\
  & Pociones para Principiantes & Profesor Severus Snape \\
  & Historia de la Magia Antigua & Profesora Bathilda Bagshot \\
  & Defensa Básica contra las Artes Oscuras & Profesor Quirinus Quirrell \\
  & Vuelo Inicial en Escoba & Madame Hooch \\
  & Estudios Muggles Introductorios & Profesor Arthur Weasley \\ \hline
  \multirow{2}{*}{Segundo Año} & Desarrollo Mágico & \\
  & Transformaciones Intermedias & Profesora Minerva McGonagall \\
  & Cuidado de Criaturas Mágicas & Profesor Rubeus Hagrid \\
  & Pociones Avanzadas & Profesora Horace Slughorn \\
  & Historia de la Magia Contemporánea & Profesor Binns (fantasma) \\
  & Defensa Intermedia contra las Artes Oscuras & Profesora Gilderoy Lockhart \\
  & Aritmancia Elemental & Profesora Septima Vector \\
  & Runas Antiguas & Profesor Bathsheda Babbling \\ \hline
  \multirow{2}{*}{Tercer Año} & Magia Avanzada & \\
  & Encantamientos Avanzados & Profesor Remus Lupin \\
  & Herbología Avanzada & Profesora Neville Longbottom \\
  & Pociones Maestras & Profesora Horace Slughorn \\
  & Historia de la Magia Medieval & Profesora Adalbert Waffling \\
  & Defensa Avanzada contra las Artes Oscuras & Profesor Alastor Moody \\
  & Adivinación Elemental & Profesora Sybill Trelawney \\
  & Astronomía Mágica & Profesor Aurora Sinistra \\ \hline
  \multirow{2}{*}{Cuarto Año} & Especialización Mágica & \\
  & Transformaciones Avanzadas & Profesora Minerva McGonagall \\
  & Cuidado de Criaturas Mágicas Exóticas & Profesor Rubeus Hagrid \\
  & Pociones Maestras Avanzadas & Profesora Horace Slughorn \\
  & Historia de la Magia Moderna & Profesor Cuthbert Binns (fantasma) \\
  & Defensa contra las Artes Oscuras Especializada & Profesor Dolores Umbridge \\
  & Estudios de Artefactos Mágicos & Profesora Bathilda Bagshot \\
  & Ocultismo & Profesor Gellert Grindelwald \\ \hline
  \multirow{2}{*}{Quinto Año} & Maestría Mágica & \\
  & Encantamientos Supremos & Profesor Filius Flitwick \\
  & Herbología Aplicada & Profesora Neville Longbottom \\
  & Pociones Maestras Maestría & Profesora Horace Slughorn \\
  & Historia de la Magia Contemporánea Avanzada & Profesora Bathilda Bagshot \\
  & Defensa Avanzada contra las Artes Oscuras Aplicada & Profesora Nymphadora Tonks \\
  & Aritmancia Avanzada & Profesora Septima Vector \\
  & Estudio de las Maldiciones & Profesor Severus Snape \\ \hline
  \multirow{2}{*}{Sexto Año} & Sabiduría Arcana & \\
  & Transformaciones Maestras & Profesor Minerva McGonagall \\
  & Cuidado de Criaturas Mágicas Legendarias & Profesor Rubeus Hagrid \\
  & Pociones Maestras Supremas & Profesor Horace Slughorn \\
  & Historia de la Magia Postmoderna & Profesora Bathilda Bagshot \\
  & Defensa Maestra contra las Artes Oscuras & Profesor Alastor Moody \\
  & Adivinación Avanzada & Profesora Sybill Trelawney \\
  & Astronomía Avanzada & Profesora Aurora Sinistra
  \end{tabular}
  \end{table}
  




%%%%%--------------------------------------
%%%%%-------------------------------------
\subsection{Modelo conceptual}
%%%%%-------------------------------------
%%%%%--------------------------------------
\begin{center}
    \includegraphics[scale = .7]{IMA/escuela/BD-ESCUELA-MO-CONCEPTUAL.png}
\end{center}


%%%%%--------------------------------------
%%%%%-------------------------------------
\subsection{Modelo relacional}
%%%%%-------------------------------------
%%%%%--------------------------------------
\begin{center}
  \includegraphics[scale = .7]{IMA/escuela/BD-ESCUELA-MO-RELACIONAL.png}
\end{center}


%%%%%--------------------------------------
%%%%%-------------------------------------
\subsection{Script de creación}
%%%%%-------------------------------------
%%%%%--------------------------------------
\begin{lstlisting}[caption={Tablas para la BdDatos}, label={lst:sql_estadios}]
    -- Solo los titulos de las tablas para escuela    
\end{lstlisting}



%%%%%--------------------------------------
%%%%%-------------------------------------
\subsection{Script de Insert}
%%%%%-------------------------------------
%%%%%--------------------------------------
\begin{itemize}
    \item[$\rightarrow$] Se deben generar 100 registros para cada tabla.
    \item[$\rightarrow$] Si para el buen funcionamiento de la base de datos se requieren más de 100 registros o
            menos de 100 registros en una tabla, se debe explicar claramente la razón, sólo en este caso
            sí se debe incluir un apartado en el reporte final.
\end{itemize}

Tablas:



%%%%%--------------------------------------
%%%%%-------------------------------------
\subsection{Funcionamiento restricciones}
%%%%%-------------------------------------
%%%%%--------------------------------------

Evidencia del funcionamiento de al menos 4 restricciones de integridad referencial.

\subsubsection*{Restricción 01}

\begin{itemize}
    \item[$\rightarrow$] Tablas involucradas en la restricción: 
    \item[$\rightarrow$] FK de la tabla que referencia y PK de la tabla referenciada: 
    \item[$\rightarrow$] Justificación del trigger de integridad referencial elegido: 
    \item[$\rightarrow$] Instrucción UPDATE o DELETE que permita evidenciar que la restricción está
    funcionando.
    \begin{verbatim}
    INSTRUCCION
    \end{verbatim}
    \item[$\rightarrow$] Captura de pantalla con el resultado de la instrucción que muestre que la restricción está
    funcionando.
    \begin{center}
        CAPTURA
    \end{center}
\end{itemize}

\subsubsection*{Restricción 01}

\begin{itemize}
    \item[$\rightarrow$] Tablas involucradas en la restricción: 
    \item[$\rightarrow$] FK de la tabla que referencia y PK de la tabla referenciada: 
    \item[$\rightarrow$] Justificación del trigger de integridad referencial elegido: 
    \item[$\rightarrow$] Instrucción UPDATE o DELETE que permita evidenciar que la restricción está
    funcionando.
    \begin{verbatim}
    INSTRUCCION
    \end{verbatim}
    \item[$\rightarrow$] Captura de pantalla con el resultado de la instrucción que muestre que la restricción está
    funcionando.
    \begin{center}
        CAPTURA
    \end{center}
\end{itemize}

\subsubsection*{Restricción 02}

\begin{itemize}
    \item[$\rightarrow$] Tablas involucradas en la restricción: 
    \item[$\rightarrow$] FK de la tabla que referencia y PK de la tabla referenciada: 
    \item[$\rightarrow$] Justificación del trigger de integridad referencial elegido: 
    \item[$\rightarrow$] Instrucción UPDATE o DELETE que permita evidenciar que la restricción está
    funcionando.
    \begin{verbatim}
    INSTRUCCION
    \end{verbatim}
    \item[$\rightarrow$] Captura de pantalla con el resultado de la instrucción que muestre que la restricción está
    funcionando.
    \begin{center}
        CAPTURA
    \end{center}
\end{itemize}

\subsubsection*{Restricción 03}

\begin{itemize}
    \item[$\rightarrow$] Tablas involucradas en la restricción: 
    \item[$\rightarrow$] FK de la tabla que referencia y PK de la tabla referenciada: 
    \item[$\rightarrow$] Justificación del trigger de integridad referencial elegido: 
    \item[$\rightarrow$] Instrucción UPDATE o DELETE que permita evidenciar que la restricción está
    funcionando.
    \begin{verbatim}
    INSTRUCCION
    \end{verbatim}
    \item[$\rightarrow$] Captura de pantalla con el resultado de la instrucción que muestre que la restricción está
    funcionando.
    \begin{center}
        CAPTURA
    \end{center}
\end{itemize}

\subsubsection*{Restricción 04}

\begin{itemize}
    \item[$\rightarrow$] Tablas involucradas en la restricción: 
    \item[$\rightarrow$] FK de la tabla que referencia y PK de la tabla referenciada: 
    \item[$\rightarrow$] Justificación del trigger de integridad referencial elegido: 
    \item[$\rightarrow$] Instrucción UPDATE o DELETE que permita evidenciar que la restricción está
    funcionando.
    \begin{verbatim}
    INSTRUCCION
    \end{verbatim}
    \item[$\rightarrow$] Captura de pantalla con el resultado de la instrucción que muestre que la restricción está
    funcionando.
    \begin{center}
        CAPTURA
    \end{center}
\end{itemize}


%%%%%--------------------------------------
%%%%%-------------------------------------
\subsection{Funcionamiento Restricciones check}
%%%%%-------------------------------------
%%%%%--------------------------------------

Evidencia del funcionamiento de al menos 3 restricciones check para “atributos” de varias
tablas.

\begin{itemize}
    \item Tabla elegida
    \item Atributo elegido
    \item Breve descripción de la restricción
    \item Instrucción para la creación de la restricción.
    \item Instrucción que permita evidenciar que la restricción esta funcionando.
    \item Captura de pantalla con el resultado de la instrucción que muestre que la restricción está
    funcionando.
\end{itemize}




%%%%%--------------------------------------
%%%%%-------------------------------------
\subsection{Creación de dominios personalizados}
%%%%%-------------------------------------
%%%%%--------------------------------------

Evidencia de la creación de al menos tres dominios personalizados. Se deben utilizar restricciones check en la creación de los tres dominios.
\begin{itemize}
    \item Tabla elegida
    \item Atributo elegido
    \item Breve descripción del dominio y de la restricción check propuesta.
    \item Instrucción para la creación del dominio personalizado.
    \item Captura de pantalla de la estructura de la tabla donde se muestre el dominio personalizado
    en uso.
\end{itemize}



%%%%%--------------------------------------
%%%%%-------------------------------------
\subsection{Restricciones para tuplas}
%%%%%-------------------------------------
%%%%%--------------------------------------

Evidencia del funcionamiento de al menos 2 restricciones para “tuplas” en diferentes tablas (Unidad 8 Integridad, tema Specifying Constraints on Tuples Using CHECK)
\begin{itemize}
    \item Tabla elegida
    \item Breve descripción de la restricción.
    \item Instrucción para la creación de la restricción.
    \item Instrucción Insert o Update que permita evidenciar que la restricción esta funcionando.
    \item Captura de pantalla con el resultado de la instrucción que muestre que la restricción está
    funcionando.
\end{itemize}



%%%%%--------------------------------------
%%%%%-------------------------------------
\subsection{Consultas}
%%%%%-------------------------------------
%%%%%--------------------------------------

Plantea 3 consultas que consideres relevantes para la base de datos propuesta. Para cada consulta planteada, incluir en el reporte los siguientes incisos:
\begin{itemize}
    \item Redacción clara de la consulta.
    \item Código en lenguaje SQL de la consulta.
    \item Ejecutar la consulta en Postgres e incluir una captura de pantalla con el resultado de la
    consulta.
\end{itemize}

%%%%%--------------------------------------
%%%%%-------------------------------------
\subsection{Vistas}
%%%%%-------------------------------------
%%%%%--------------------------------------

Plantea 3 vistas que consideres relevantes para la base de datos propuesta. Para cada vista planteada, incluir en el reporte los siguientes incisos:
\begin{itemize}
    \item Redacción clara de la vista planteada.
    \item Código en lenguaje SQL que permita crear la vista solicitada.
    \item Ejecutar el código para la creación de la vista en Postgres e incluir una captura de pantalla
    con la vista creada satisfactoriamente.
    \item Incluir un ejemplo que los evaluadores puedan ejecutar para verificar el funcionamiento
    de las vistas.
\end{itemize}